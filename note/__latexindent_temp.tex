% This file was created with JabRef 2.9.2.
% Encoding: utf8

@MISC{pgm_course,
  author = {\mbox{Stanford University}},
  title = {Probabilistic Graphical Models [Электронный ресурс]},
  howpublished = {Электронные данные},
  note = {Режим доступа: \url{https://www.coursera.org/course/pgm}. --- Дата
	доступа: 08.05.2013},
  owner = {mstyura},
  timestamp = {2013.05.08}
}

@BOOK{albahari_2012_en,
  title = {C\# 5.0 in a Nutshell},
  publisher = {O’Reilly Media, Inc},
  year = {2012},
  author = {Joseph Albahari and Ben Albahari},
  pages = {1062},
  edition = {5-th},
  month = {June},
  owner = {mstyura},
  timestamp = {2013.03.21}
}

@INPROCEEDINGS{beinlich1989alarm,
  author = {Beinlich, I. A. and Suermondt, H. J. and Chavez, R. M. and Cooper,
	G. F.},
  title = {{The ALARM Monitoring System: A Case Study with Two Probabilistic
	Inference Techniques for Belief Networks}},
  booktitle = {Second European Conference on Artificial Intelligence in Medicine},
  year = {1989},
  editor = {Hunter, J. and Cookson, J. and Wyatt, J.},
  volume = {38},
  pages = {247--256},
  address = {Berlin, Germany},
  publisher = {Springer-Verlag},
  citeulike-article-id = {1616436},
  keywords = {alarm, bayesian\_network},
  posted-at = {2007-09-03 16:53:17},
  priority = {0}
}

@MISC{Chickering96learningbayesian,
  author = {David Maxwell Chickering},
  title = {Learning Bayesian Networks is NP-Complete},
  year = {1996},
  language = {english}
}

@ARTICLE{Chow68approximatingdiscrete,
  author = {C. I. Chow and Sexior Member and C. N. Liu},
  title = {Approximating discrete probability distributions with dependence
	trees},
  journal = {IEEE Transactions on Information Theory},
  year = {1968},
  volume = {14},
  pages = {462--467}
}

@INPROCEEDINGS{Cooper1991,
  author = {Cooper, Gregory F. and Herskovits, Edward},
  title = {A Bayesian method for constructing Bayesian belief networks from
	databases},
  booktitle = {Proceedings of the Seventh conference on Uncertainty in Artificial
	Intelligence},
  year = {1991},
  series = {UAI'91},
  pages = {86--94},
  address = {San Francisco, CA, USA},
  publisher = {Morgan Kaufmann Publishers Inc.},
  acmid = {2100674},
  isbn = {1-55860-203-8},
  location = {Los Angeles, CA},
  numpages = {9},
  url = {http://dl.acm.org/citation.cfm?id=2100662.2100674}
}

@BOOK{crundwell_2008,
  title = {Finance for engineers evaluation and funding of capital projects},
  publisher = {Springer},
  year = {2008},
  author = {F. K. Crundwell},
  pages = {622},
  address = {London},
  owner = {mstyura},
  timestamp = {2013.03.14}
}

@INPROCEEDINGS{Grunwald05atutorial,
  author = {Peter Grünwald},
  title = {A Tutorial Introduction to the Minimum Description Length Principle},
  booktitle = {Advances in Minimum Description Length: Theory and Applications},
  year = {2005},
  publisher = {MIT Press}
}

@BOOK{harrison_1997,
  title = {Introduction to Functional Programming},
  year = {1997},
  author = {Jonh Harrison},
  pages = {168},
  address = {University of Cambridge},
  month = {December},
  language = {english},
  owner = {mstyura},
  timestamp = {2013.03.22}
}

@MISC{itechart_2013,
  author = {iTechArt},
  title = {Company profile [Электронный ресурс]},
  howpublished = {Электронные данные},
  note = {Режим доступа: http://www.itechart.com/docs/companyprofile.pdf},
  address = {Iselin, NJ 08830 USA},
  language = {english},
  owner = {mstyura},
  timestamp = {2013.03.21}
}

@BOOK{Koller_2009,
  title = {Probabilistic Graphical Models: Principles and Techniques},
  publisher = {MIT Press},
  year = {2009},
  author = {D. Koller and N. Friedman},
  pages = {1270}
}

@ARTICLE{Lam94learningbayesian,
  author = {Wai Lam and Fahiem Bacchus},
  title = {Learning Bayesian belief networks: An approach based on the MDL principle},
  journal = {Computational Intelligence},
  year = {1994},
  volume = {10},
  pages = {269--293},
  language = {english}
}

@ARTICLE{Lauritzen_Spiegelhalter88,
  author = {Lauritzen, S. L. and Spiegelhalter, D. J.},
  title = {{Local Computations with Probabilities on Graphical Structures and
	Their Application to Expert Systems}},
  journal = {Journal of the Royal Statistical Society. Series B (Methodological)},
  year = {1988},
  volume = {50},
  number = {2},
  abstract = {{A causal network is used in a number of areas as a depiction of patterns
	of `influence' among sets of variables. In expert systems it is common
	to perform `inference' by means of local computations on such large
	but sparse networks. In general, non-probabilistic methods are used
	to handle uncertainty when propagating the effects of evidence, and
	it has appeared that exact probabilistic methods are not computationally
	feasible. Motivated by an application in electromyography, we counter
	this claim by exploiting a range of local representations for the
	joint probability distribution, combined with topological changes
	to the original network termed `marrying' and `filling-in'. The resulting
	structure allows efficient algorithms for transfer between representations,
	providing rapid absorption and propagation of evidence. The scheme
	is first illustrated on a small, fictitious but challenging example,
	and the underlying theory and computational aspects are then discussed.}},
  citeulike-article-id = {236694},
  citeulike-linkout-0 = {http://dx.doi.org/10.2307/2345762},
  citeulike-linkout-1 = {http://www.jstor.org/stable/2345762},
  doi = {10.2307/2345762},
  issn = {00359246},
  keywords = {bayesian-networks},
  posted-at = {2005-06-25 05:09:56},
  priority = {1},
  publisher = {Blackwell Publishing for the Royal Statistical Society},
  url = {http://dx.doi.org/10.2307/2345762}
}

@INPROCEEDINGS{Rebane87,
  author = {George Rebane and Judea Pearl},
  title = {The Recovery of Causal Poly-Trees From Statistical Data},
  booktitle = {Uncertainty in Artificial Intelligence 3 Annual Conference on Uncertainty
	in Artificial Intelligence (UAI-87)},
  year = {1987},
  pages = {175--182},
  address = {Amsterdam, NL},
  publisher = {Elsevier Science}
}

@ELECTRONIC{fsdg_2010,
  author = {Microsoft Research and Microsoft Developer Division},
  month = {April},
  year = {2010},
  title = {Draft F\# Component Design Guidelines},
  language = {english},
  organization = {Microsoft},
  owner = {mstyura},
  timestamp = {2013.03.21}
}

@BOOK{richter_2012_en,
  title = {CLR via C\#},
  publisher = {Microsoft Press},
  year = {2012},
  author = {Jeffrey Richter},
  pages = {896},
  series = {Microsoft, Developer Reference},
  address = {One Microsoft Way, Redmond, Washington 98052-6399},
  edition = {4-th},
  isbn = {0735667454,9780735667457},
  owner = {mstyura},
  timestamp = {2013.03.20}
}

@ARTICLE{linkedIn,
  title={The virtual geographies of social networks: a comparative analysis of Facebook, LinkedIn and ASmallWorld},
  author={Papacharissi, Zizi},
  journal={New media \& society},
  volume={11},
  number={1-2},
  pages={199--220},
  year={2009},
  publisher={SAGE Publications Sage UK: London, England}
}

@ARTICLE{facebook,
  title={The benefits of Facebook “friends:” Social capital and college students’ use of online social network sites},
  author={Ellison, Nicole B and Steinfield, Charles and Lampe, Cliff},
  journal={Journal of computer-mediated communication},
  volume={12},
  number={4},
  pages={1143--1168},
  year={2007},
  publisher={Oxford University Press Oxford, UK}
}

@ARTICLE{youtube,
  title={YouTube},
  author={YouTube, LLC},
  journal={Retrieved},
  volume={27},
  pages={2011},
  year={2011}
}

@ARTICLE{slack,
  title={Is slack good or bad for innovation?},
  author={Nohria, Nitin and Gulati, Ranjay},
  journal={Academy of management Journal},
  volume={39},
  number={5},
  pages={1245--1264},
  year={1996},
  publisher={Academy of Management Briarcliff Manor, NY 10510}
}

@INPROCEEDINGS{gitHub,
  title={Social coding in GitHub: transparency and collaboration in an open software repository},
  author={Dabbish, Laura and Stuart, Colleen and Tsay, Jason and Herbsleb, Jim},
  booktitle={Proceedings of the ACM 2012 conference on computer supported cooperative work},
  pages={1277--1286},
  year={2012},
  organization={ACM}
}

@ARTICLE{verticalSN,
  title={Social networks, social support and social capital: The experiences of recent Polish migrants in London},
  author={Ryan, Louise and Sales, Rosemary and Tilki, Mary and Siara, Bernadetta},
  journal={Sociology},
  volume={42},
  number={4},
  pages={672--690},
  year={2008},
  publisher={Sage Publications Sage UK: London, England}
}

@ARTICLE{ingate,
  title={Availability of social networks as marketing instrument in esthetic surgery},
  author={Khramtsova, NI and Plaksin, SA and Lebedeva, TM and Ens, MA and Belyakova, OS},
  journal={Perm Medical Journal},
  volume={35},
  number={5},
  pages={70--74},
  year={2018}
}

@INPROCEEDINGS{Spohn94onthe,
  author = {Wolfgang Spohn and Abteilung Philosophie and D Bielefeld},
  title = {On the Properties of Conditional Independence},
  booktitle = {Suppes: Scientific philosopher},
  year = {1994},
  pages = {173--194},
  publisher = {Kluwer}
}

@INPROCEEDINGS{Suzuki93,
  author = {Joe Suzuki },
  title = {A Construction of Bayesian Networks from Databases Based on an MDL
	Principle},
  booktitle = {Proceedings of the Ninth Conference Annual Conference on Uncertainty
	in Artificial Intelligence (UAI-93)},
  year = {1993},
  pages = {266--273},
  address = {San Francisco, CA},
  publisher = {Morgan Kaufmann}
}

@BOOK{sicp_2006_ru,
  title = {Структура и интерпретация компьютерных программ},
  publisher = {Добросвет},
  year = {2006},
  author = {Харольд Абельсон and Джеральд Джей Сассман and Джули Сассман},
  pages = {608},
  language = {russian},
  owner = {mstyura},
  timestamp = {2013.03.20}
}

@BOOK{volosevich_cs_2011,
  title = {Язык C\# и основы платформы .NET: Учебно"=метод. пособие по курсу
	<<Избранные главы информатики>> для студ. спец. \mbox{I-31~03~04}
	<<Информатика>> всех форм обуч.},
  year = {2006},
  author = {А. А. Волосевич},
  pages = {60},
  address = {Минск},
  language = {russian},
  owner = {mstyura},
  timestamp = {2013.03.21}
}

@BOOK{guk_1999,
  title = {Процессоры Pentium II, Pentium Pro и просто Pentium},
  publisher = {Питер Ком},
  year = {1999},
  author = {М. Гук},
  pages = {288},
  language = {russian},
  owner = {mstyura},
  timestamp = {2013.02.20}
}

@ARTICLE{pfp_funelem_issue3_2009,
  author = {Е. Кирпичёв},
  title = {Элементы функциональных языков},
  journal = {Практика функционального программирования},
  year = {2009},
  pages = {196},
  number = {3},
  month = {Декабрь},
  language = {russian},
  owner = {mstyura},
  timestamp = {2013.03.22}
}

@BOOK{kluev_1989,
  title = {Технические средства диагностирования : справочник},
  publisher = {Машиностроение},
  year = {1989},
  author = {В.В. Клюев1 and В.В. Клюев2 and В.В. Клюев3 and В.В. Клюев4 and В.В.
	Клюев5},
  pages = {672},
  address = {М.},
  language = {russian},
  owner = {mstyura},
  timestamp = {2013.02.20}
}

@BOOK{kulezin_2004,
  title = {Современные семейства ПЛИС фирмы Xilinx : справ. пособие},
  publisher = {Горячая линия–Телеком},
  year = {2004},
  author = {М. О. Кузелин and Д. А. Кнышев and В. Ю. Зотов},
  pages = {440},
  address = {М.},
  language = {russian},
  owner = {mstyura},
  timestamp = {2013.02.20}
}

@ARTICLE{fsharp_pfp_issue_5,
  author = {Е. Лазин and М. Моисеев and Д. Сорокин},
  title = {Введение в F\#},
  journal = {Практика функционального программирования},
  year = {2010},
  pages = {149},
  number = {5},
  language = {russian},
  owner = {mstyura},
  timestamp = {2013.03.21}
}

@BOOK{mcconnell_2005,
  title = {Совершенный код. Мастер"=класс / Пер. с англ.},
  publisher = {Издательско"=торговый дом <<Русская редакция>>},
  year = {2005},
  author = {С. Макконнелл},
  pages = {896},
  address = {СПб.},
  language = {russian},
  owner = {mstyura},
  timestamp = {2013.03.18}
}

@BOOK{marchenko_2007,
  title = {Основы программирования на C\# 2.0},
  publisher = {БИНОМ. Лаборатория знаний, Интернет"=университет информационных технологий
	"--- ИНТУИТ.ру},
  year = {2007},
  author = {А. Л. Марченко},
  pages = {552},
  language = {russian},
  owner = {mstyura},
  timestamp = {2013.03.20}
}

@BOOK{michailov_2011,
  title = {Пожарная безопасность в офисе},
  publisher = {Альфа-Пресс},
  year = {2011},
  author = {Ю. М. Михайлов},
  language = {russian},
  owner = {mstyura},
  timestamp = {2013.02.26}
}

@BOOK{michnuk_2009,
  title = {Охрана труда : учебник [утв. МО РБ]},
  publisher = {ИВЦ Минфина},
  year = {2009},
  author = {Т. Ф. Михнюк},
  pages = {345},
  address = {Минск},
  language = {russian},
  owner = {mstyura},
  timestamp = {2013.02.26}
}

@BOOK{yashin_2009,
  title = {Охрана труда, экологическая безопасность, энергосбережение : метод.
	пособие по выполнению дипломных проектов (работ)},
  publisher = {БГУИР},
  year = {2009},
  editor = {Т. Ф. Михнюка},
  author = {Т. Ф. Михнюк AND И. С. Асаенок AND И. И. Кирвель AND А. И. Навоша
	AND К. Д. Яшин},
  pages = {36},
  address = {Минск},
  language = {russian},
  owner = {mstyura},
  timestamp = {2013.02.26}
}

@Book{palitsyn,
  title     = {Технико-экономическое обоснование дипломных проектов: Метод. пособие для студ. всех спец. БГУИР. В 4-х ч. Ч. 4: Проекты программного обеспечения},
  publisher = {БГУИР},
  year      = {2006},
  author    = {В.А. Палицын},
  address   = {Мн},
  isbn      = {9854449068},
  note      = {76~с.},
}

@ARTICLE{nosenko,
  title={Методические указания по технико-экономическому обоснованию дипломных проектов инженерного профиля},
  author={Носенко, АА},
  journal={Мн., БГУИР},
  year={2010}
}

@MISC{labour_calendar,
  author       = {Марина Тимина},
  title        = {О рабочем времени в 2019 году [Электронный ресурс]},
  howpublished = {Электронные данные},
  note         = {Режим доступа: https://ilex.by/o-rabochem-vremeni-v-2019-godu/. --- Дата доступа: 08.04.19},
}

@MISC{incomeTaxRate,
  owner = {gb},
  title = {Налог на прибыль в 2019 году [Электронный ресурс]},
  howpublished = {Электронные данные},
  note = {Режим доступа: https://www.gb.by/aktual/nalogooblozhenie/nalog-na-pribyl-v-2019-godu. --- Дата доступа: 08.04.19},
  timestamp = {2018.03.22},
}

@BOOK{Morozov_2011,
  title = {Подготовка документов в издательской системе Латех},
  publisher = {ЯрГУ им. П.Г. Демидова},
  year = {2011},
  author = {Д.К. Морозов and А.Я. Пархоменко},
  language = {russian},
  owner = {mstyura},
  timestamp = {2013.02.20}
}

@BOOK{palicyn_2006,
  title = {Технико-экономическое обоснование дипломных проектов: Метод. пособие
	для студ. всех спец. БГУИР. В 4-х ч. Ч. 4: Проекты программного обеспечения},
  publisher = {БГУИР},
  year = {2006},
  author = {В. А. Палицын},
  pages = {76},
  address = {Минск},
  language = {russian},
  owner = {mstyura},
  timestamp = {2013.03.03}
}

@BOOK{richter_2012_ru,
  title = {CLR via С\#. Программирование на платформе Microsoft .NET Framework
	4.0 на языке С\#. 3-е изд.},
  publisher = {Питер},
  year = {2012},
  author = {Дж. Рихтер},
  pages = {928},
  address = {СПб.},
  language = {russian},
  owner = {mstyura},
  timestamp = {2013.03.20}
}

@BOOK{richter_2007_ru,
  title = {CLR via C\#. Программирование на платформе Microsoft .NET Framework
	2.0 на языке C\#},
  publisher = {Питер, Русская Редакция},
  year = {2007},
  author = {Джеффри Рихтер},
  pages = {656},
  address = {СПб.},
  edition = {2-е},
  language = {russian},
  owner = {mstyura},
  timestamp = {2013.03.20}
}

@BOOK{sinilov_2010,
  title = {Системы охранной, пожарной и охранно-пожарной сигнализации : учебник
	для нач. проф. образования},
  publisher = {Издательский центр <<Академия>>},
  year = {2010},
  author = {В. Г. Синилов},
  pages = {512},
  address = {М.},
  edition = {5-е},
  language = {russian},
  owner = {mstyura},
  timestamp = {2013.03.15}
}

@ARTICLE{terentyev_2006,
  author = {А. Н. Терентьев and П. И. Бидюк},
  title = {Эвристический метод построения байесовых сетей},
  journal = {Математические машины и системы},
  year = {2006},
  number = {3},
  language = {russian},
  owner = {mstyura},
  timestamp = {2013.05.07}
}

@CONFERENCE{terehov_2003,
  author = {С. А. Терехов},
  title = {Введение в байесовы сети},
  booktitle = {Научная cессия МИФИ--2003. V Всероссийская научно-техническая Конференция
	<<Нейроинформатика--2003>>: Лекции по нейроинформатике. Часть~1.},
  year = {2003},
  pages = {188},
  address = {М.},
  publisher = {МИФИ},
  language = {russian},
  owner = {mstyura},
  timestamp = {2013.03.30}
}

@BOOK{sharovar_1979,
  title = {Устройства и системы пожарной сигнализации},
  publisher = {Стройиздат},
  year = {1979},
  author = {Ф. И. Шаровар},
  pages = {271},
  address = {М.},
  language = {russian},
  owner = {mstyura},
  timestamp = {2013.03.14}
}

@BOOK{microproc_1988,
  title = {Микропроцессоры и микропроцессорные комплекты интегральных микросхем.
	В 2 т.},
  publisher = {Радио и связь},
  year = {1988.},
  editor = {В. А. Шахнова},
  volume = {1},
  pages = {368},
  address = {М.},
  language = {russian},
  owner = {mstyura},
  timestamp = {2013.02.20}
}

@MISC{belcalendar_2013,
  title = {Календарь праздников на 2013 год для Беларуси [Электронный ресурс]},
  howpublished = {Электронные данные},
  note = {Режим доступа: \url{http://calendar.by/2013/\#bkm}. --- Дата доступа:
	05.03.2013},
  language = {russian},
  owner = {mstyura},
  timestamp = {2013.03.05}
}

@MISC{cite_webpage,
  title = {How can I use BibTeX to cite a web page? [Электронный ресурс]},
  howpublished = {Электронные данные},
  note = {Режим доступа: http://tex.stackexchange.com/questions/3587/how-can-i-use-bibtex-to-cite-a-web-page},
  language = {english},
  owner = {mstyura},
  timestamp = {2013.02.20}
}

@MISC{csharp_wiki_2013_ru,
  title = {C Sharp [Электронный ресурс]},
  howpublished = {Электронные данные},
  note = {Режим доступа: \url{http://ru.wikipedia.org/wiki/C\_Sharp}. --- Дата
	доступа: 22.03.2013},
  owner = {mstyura},
  timestamp = {2013.03.22}
}

@STANDARD{ecma_335,
  title = {Common Language Infrastructure (CLI). Partitions I to VI.},
  organization = {ECMA International},
  language = {english},
  month = {June},
  year = {2012},
  url = {http://www.ecma-international.org/publications/standards/Ecma-335.htm},
  edition = {6-th},
  owner = {mstyura},
  timestamp = {2013.03.19}
}

@MISC{refinancingTax,
  title = {Ставка рефинансирования Национального банка Республики Беларусь [Электронный ресурс]},
  howpublished = {Электронные данные},
  note = {Режим доступа: https://www.gb.by/aktual/nalogooblozhenie/nalog-na-pribyl-v-2019-godu. --- Дата
  доступа: 08.04.19},
  owner = {Беларусьбанк},
  timestamp = {2018.03.22},
}
}

