% Содержимое данного документа позаимсвовано из Приложения Ж из документа http://www.bsuir.by/m/12_113415_1_66883.pdf

\thispagestyle{empty}

\begin{singlespace}

{\small
  \begin{center}
    \begin{minipage}{0.9\textwidth}
      \begin{center}
        {\normalsize РЕЦЕНЗИЯ}\\[0.2cm]
        на дипломный проект студента инженерно-экономического факультета Учреждения образования <<Белорусский государственный университет информатики и радиоэлектроники>>\\
        Покосенко Владислава Юрьевича \\
        на тему: <<Тематические социальные сети и кроссплатформенное мобильное приложение для мотоциклистов>>
      \end{center}
    \end{minipage}\\
  \end{center}

Студент Покосенко В.Ю. выполнил дипломный проект, который оформлен в виде пояснительной записки на 84 страницы, четырех приложений и графического материала на шести листах (два чертежа и четыре плаката).

Дипломный проект посвящен вопросу создания тематических социальных сететй с применением кроссплатформенного мобильного приложения. Актуальность темы обусловлена тем, что последнее время растет популярность социальных сетей данного вида, а кроссплатформенные решения в несколько раз сокращают затраты на разработку мобильных приложений.

Пояснительная записка построена логично и последовательно отражает все этапы разработки в соответствии с календарным планом. По каждому разделу и в целом по дипломному проекту приведены аргументированные выводы.

В аналитической части описана роль социальных сетей в современном обществе; отдельное внимание выделено тематическим социальным сетям, методам и походам мобильной разработки, аргументиован выбор кроссплатформенного решения.

В проектной части проведен анализ процесса организации совместной поездки мотоциклистов, разработана спецификация функциональных требований к приложению и инфологическая модель базы данных.

В части реализации программного средства представлено мобильное приложение для социальной сети мотоциклистов. Приложение обладает удобным многофункциональным интерфейсом, работает на обоих платформах (IOS и Android). Подробно описана релаизация серверной и клиенской части приложения.

В рецензируемой работе осуществлено грамотное технико-экономическое обоснование целесообразности внедрения разработанного программного средства.
Приведенные расчеты и программное обеспечение свидетельствуют о глубоких знаниях студента Покосенко~В.\,Ю. в области проектирования подобных приложений, умении работать с технической литературой и применять на практике наиболее рациональные и современные решения.
Считаю, что разработанное мобильное приложение может получить большую популярность у пользователей, интересующихся мотоциклетной тематикой.

Замечания:
\begin{itemize}
  \item отсутствуют статистические данные о пользователях социальных сетей за последние пару лет;
  \item нет описания клиент-серверного взаимодействия в реальном времени;
  \item отсутствует обзор и сравнение с другим популярным кроссплатформенным решением Flatter.
\end{itemize}

В целом дипломный проект выполнен технически грамотно, в полном соответствии с техническим заданием на проектирование и заслуживает оценки «\underline{ {} {} {} {} }» баллов, а Покосенко~В.\,Ю. "--- присвоения квалификации «инженер-экономист-программист».


  \vfill
  \noindent
  \vspace*{1cm}
  \begin{minipage}{0.29\textwidth}
    \begin{flushleft}
      Рецензент:\\
      канд. техн. наук, доцент\\
      кафедры ПОСТ УО
    \end{flushleft}
  \end{minipage}
  \begin{minipage}{0.58 \textwidth}
    \begin{flushleft}
    \hspace*{0.5cm}\underline{\hspace*{4cm}} В.\,А.~Новиков \\
   
    \end{flushleft}
  \end{minipage}
}

\end{singlespace}
\clearpage