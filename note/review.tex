% Содержимое данного документа позаимсвовано из Приложения Ж из документа http://www.bsuir.by/m/12_113415_1_66883.pdf

\thispagestyle{empty}

\begin{singlespace}

{\small
  \begin{center}
    \begin{minipage}{0.9\textwidth}
      \begin{center}
        {\normalsize РЕЦЕНЗИЯ}\\[0.2cm]
        на дипломный проект студента факультета компьютерных систем и сетей Учреждения образования <<Белорусский государственный университет информатики и радиоэлектроники>>\\
        Покосенко Владислава Юрьевича \\
        на тему: <<Устройство квантово-криптографического закрытия информации>>
      \end{center}
    \end{minipage}\\
  \end{center}

Дипломный проект студента Радевича С. И. состоит из шести листов графического материала и 84 страницы пояснительной записки.

Тема проекта является актуальной и посвящена разработке кроссплатформенного мобильного приложения, предназначенного для тематической социальной сети. 
Разработка данного устройства обусловлена необходимостью создания средств связи, надѐжно защищенных от несанкционированного доступа.

Пояснительная записка построена логично и последовательно отражает все этапы разработки в соответствии с календарным планом.

В пояснительной записке достаточно полно сделан обзор современных криптографических методов генерации секретного ключа, четко изложены методы генерации секретного
ключа в квантовой криптографии.
Разработаны схема продвижения информации в квантовой криптографии, конструкции передающего и принимающего устройств; выбраны источник и детектор единичных фотонов; предложен механизм, управляющий поляризацией отправляемых в канал связи фотонов, который основан на использовании биморфной пьезоэлектрической балки в качестве микроисполнительного устройства. 
Произведен выбор метода передачи двоичных сигналов, разработаны алгоритмы функционирования, схемы структурные и принципиальные.
В проекте приведен глубокий аналитический обзор научно"=технической литературы, где рассмотрены все вопросы, касающиеся темы проекта.
Приведенные расчеты и программное обеспечение свидетельствуют о глубоких знаниях студента Покосенко~В.\,Ю. в области проектирования подобных систем, умении работать с технической литературой и применять на практике наиболее рациональные решения.

По каждому разделу и в целом по дипломному проекту приведены аргументированные выводы.

Пояснительная записка и графический материал оформлены аккуратно и в соответствии с требованиями ЕСКД.
Считаю, что разработанное мобильное приложение может получить большую популярность у пользователей.

Замечания:
\begin{itemize}
  \item отсутствует статистика пользователей в социальных сетях за последние пару лет;
  \item нет описания клиент-серверного взаимодействия в реальном времени;
  \item нет описания влияния кроссплатформенности на жкономическую выгоду реализации приложения.
\end{itemize}

В целом дипломный проект выполнен технически грамотно, в полном соответствии с техническим заданием на проектирование и заслуживает оценки десять баллов, а диплом
ник Покосенко~В.\,Ю. "--- присвоения квалификации инженер-программист-экономист.

  \vfill
  \noindent
  \begin{minipage}{0.4\textwidth}
    \begin{flushleft}
      Рецензент:\\
      канд. техн. наук, профессор\\
      кафедры ИТАС БГУИР
    \end{flushleft}
  \end{minipage}
  \begin{minipage}{0.58\textwidth}
    \begin{flushright}
    \underline{\hspace*{3cm}}\hspace*{0.5cm}\underline{\hspace*{2cm}} В.\,В.~Новиков \\
    Дата\hspace*{6.5cm}
    \end{flushright}
  \end{minipage}
}

\end{singlespace}
\clearpage