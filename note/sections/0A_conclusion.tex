\sectioncentered*{Заключение}
\addcontentsline{toc}{section}{Заключение}

Предметной областью данного дипломного проекта является тематическая социальная сеть мотоциклистов. 
%Был проведен поиск существующих программные средств этого рода, по его результатам был сделан вывод о несуществовании полных аналогов. 
Было предложено программное средство, которое должно объеденять функционал обычной социальной сети и функционал, необходимый для мотоциклистов.

На основании проведенного анализа предметной области были выдвинуты требования к программному средству. В качестве технологий разработки были выбраны наиболее современные существующие на данный момент средства, широко применяемые в индустрии. 
%Спроектированное программное средство было успешно протестировано на соответствие спецификации функциональных требований.
%Уже исходя только из анализа предметной области и факта несуществования полных аналогов можно было сделать вывод о целесообразности проектирования и разработки программной системы.
Результаты, полученные в ходе выполнения технико-экономического обоснования только подтвердили данный вывод.

Разработано программное средство, целевой платформой которого является кроссплатформенное мобильное приложение, поддерживающее следующие функции:
\begin{itemize}
	\item создание и управление профилем пользователя;
	\item возможность создания публикаций с фото и видеоматериалами;
	\item добавление хеш-тегов к публикациям для их поиска другими пользователями;
	\item обмен сообщениями между пользователями приложения;
	\item создание и управление гаражом;
	\item ведение личного блога;
	\item лента публикаций;
	\item управление друзьями и подписчиками;
	\item возможность подачи сигнала SOS c оповещением друзей;
	\item установка меток предупреждения на карте;
	\item отправка маячка для совместной поездки;
	\item настройки приватности.
\end{itemize}

Следующая основная цель -- популяризация приложения среди мотоциклистов. Параллельно с этим будет производиться дальнейшая его разработка. Будет внедрена поддержка построения маршрута на карте. Кроме того, будут внедряться новые функции, например: анализ понравившихся публикаций и генерирование похожих в ленту пользователя. 