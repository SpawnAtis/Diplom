\subsection{Тематические социальные сети}
\label{sec:analysis:analogues}

Социальная сеть - это отличный инструмент для общения, поддержки личных связей, обмена информацией. И когда-то социальные сети полностью подходили под это описание.
Сейчас все иначе, сейчас социальная сеть -- это медийный <<журнал>>, хаотичная смесь из информационно-развлекательного контента и рекламы. В основном пользователь не создает контент, он его потребляет в этом <<журнальном» формате>>. Приэтом существует огромное количество «рубрик» - групп по интересам, где можно найти информацию на любой вкус и цвет. В связи с этим личное общение уходит на второй план, и, с молчаливого согласия самого пользователя, он превращается в «подписчика». В отличие от того, какой контент ему интереснее, он выбирает Facebook или <<ВКонтакте>>, либо любую другую социальную сеть.

Вслед за сетями общего типа, стали развиваться \emph{тематические сети}, которые использовали тот же функционал, но в более ограниченной, конкретной нише. Сейчас этот процесс перешел в более активную стадию. Можно уже находить и регистрироваться в социальных сетях для туристов, спортсменов, меломанов, книголюбов, фотографов, ученых, политиков и т.д. Но, в мире так много таких ниш, что даже, казалось бы, популярные и то до сих пор не заполнены. 

Интересно, что практически все тематические сети -- это проекты глобального уровня, они не ориентированы только на одну определенную страну. Можно смело утверждать, что стадия, на которой создавались глобальные проекты, завершается. На смену ей идет стадия, на которой создаются тематические социальные сети на рынках локальных. И, как правило, зачастую они представляют собой клоны уже существующих популярных проектов. На смену количественному насыщению рынка тематическими проектами, идет развитие качественной стороны сетей. Таким образом, повторилась история развития общих сетей: на смену количественной пришла качественная составляющая конкуренции.

Современный пользователь вовсе не желает получать контент редакционный, который для него отобрали без учета его мнения. Он хочет не только сам этот процесс контролировать, но и участвовать в нем непосредственно, лично. Это и есть главная причина, по которой создают тематическую социальную сеть для каждой значимой тематики. Эти сети перетягивают на себя большинство пользователей из общего числа целевой аудитории. Такие сообщества основываются на взаимодействии между тематическим контентом и специфической коммуникацией пользователей. Причем значительное место в них отводится различным дополнительным сервисам (например, в области геолокации или электронной коммерции).



