\subsection{Кроссплатформенность приложений} 
\label{sec:analysis:literature:crossplatform}

Как несколько десятилетий назад, так и в настоящее время выбор платформы является серьезным ограничением для всех последующих этапов разработки. Однако уже начали появляться технологии, которые позволяют использовать однажды написанный код на многих платформах. Сегодня все больше приложений создается сразу для нескольких платформ, а приложения, созданные изначально для одной платформы, активно адаптируются под другие \cite{crossplatform}. В случае мобильных приложений, код такого приложения работает как под IOS, так и под Android платформы. Разработчик, изучающий какую-либо из таких технологий, получает конкурентное преимущество, поскольку за счет расширения количества платформ расширяется круг задач, над которыми он может работать. Поэтому кроссплатформенность -- реальная или потенциальная -- является одним из факторов, который необходимо учитывать при выборе технологий реализации проекта.
