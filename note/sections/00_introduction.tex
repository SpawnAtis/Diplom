\sectioncentered*{Определения и сокращения}
\label{sec:definitions}

В настоящей пояснительной записке применяются следующие определения и сокращения.
\\

\emph{Спецификация} -- документ, который желательно полно, точно и верифицируемо определяет требования, дизайн, поведение или другие характеристики компонента или системы, и, часто, инструкции для контроля выполнения этих требований \cite{istqb_specification}.

% \emph{Веб-приложение} -- клиент-серверное приложение, в котором клиентом выступает браузер, а сервером -- веб-сервер.

\emph{Кроссплатформенность} -- способность программного обеспечения работать более чем на одной аппаратной платформе и (или) операционной системе.

\emph{Нативное программное средство} -- программное средство, специфичное для какой-либо платформы \cite{habr_crossplatform}.

\emph{Лайк} -- показатель отношения пользователей к сообщению в соцсетях, сайту, записи в блоге, сайту в поисковой выдаче или контекстному объявлению.

\emph{Хештег} -- ключевое слово или несколько слов сообщения, тег (пометка), используемый в микроблогах и социальных сетях, облегчающий поиск сообщений по теме или содержанию и начинающийся со знака решётки.

\emph{Push-уведомление} -- это короткое сообщение, которое веб-ресурс рассылает своим подписчикам на компьютеры и мобильные устройства \cite{pushNotification}.
\\

ВУЗ -- высшее учебное заведение.

ПС -- программное средство.

ПО -- программное обеспечение.

БД -- база данных.

СУБД -- система управления базами данных.

ЯП -- язык программирования.

API -- application programming interface (сетевой программный интерфейс).

UI -- user interface (пользовательский интерфейс).

SOS -- международный сигнал бедствия.

ТЭО -- технико-экономическое обоснование.


\sectioncentered*{Введение}
\addcontentsline{toc}{section}{Введение}
\label{sec:introduction}

Социальные медиа — это большой ажиотаж, когда дело касается
бизнеса, развлечений и средств массовой информации. На первых этапах
своего становления этим пользовалось только молодёжь, но теперь
используют все. Многие люди не понимают, какое влияние социальные
медиа оказывают на людей, начиная от возможности трудоустройства, и
заканчивая влиянием на мнение всей общественности.

Частью того, что делает социальные сети зависимыми, является то, что
люди могут общаться с миром простым и удобным способом; следовательно,
это создает отношение пользователя к платформе социальных сетей, которую
компании могут использовать для рекламы, и даже служит сетевым
инструментом, который позволяет предприятиям или даже
правоохранительным органам получать критически важную информацию о
конкретных людях, что делает социальные сети чрезвычайно полезным
аспектом онлайн-культуры.

В социальных сетях так много разнообразия, что для большинства
людей они стали главными центрами развлечений, для некоторых –
основным способом общения, тем самым побуждая компании пытаться
нажиться на помешательстве людей.

Социальные сети изменили способы общения и ведения бизнеса. Люди
со всего мира используют социальные сети для общения с семьей и
друзьями. Существуют web-сайты, такие как LinkedIn~\cite{linkedIn}, где пользователи
могут размещать свои резюме и связываться с работодателями. Работодатели
могут делать тоже самое, включая в свои стратегии найма использование
данного ресурса, чтобы найти достойных, высококвалифицированных
работников. Помимо самих социальных сетей, как инструмента общения,
например Facebook предлагает пользователям глобальный доступ к рекламе и
продвижению товаров и услуг~\cite{facebook}.

Различные платформы имеют свои сильные и слабые стороны в
зависимости от настроек безопасности или конфиденциальности, их
алгоритмов и пользовательских интерфейсов. Например, на YouTube~\cite{youtube}
постоянно появляются новые алгоритмы, которые делают одни видео
популярными, а другие исчезают в тени. Slack~\cite{slack} имеет огромную
популярность у IT-компаний, т.к. имеет удобную и настраиваемую
интеграцию со сторонними ресурсами, например с GitHub~\cite{gitHub}.

Для рекламодателей социальные сети предоставляют уникальные
возможности непосредственного контакта с потребителями. Ежедневно
миллионы пользователей ведут беседы о компаниях, их товарах и услугах,
делясь своим мнением и впечатлениями. В результате отдельно взятый
участник сетевого сообщества может испортить (или наоборот улучшить)
репутацию компании с многомиллионным оборотом.

Это то, что разделяет платформы социальных сетей и делает их
полезными для разных случаев. Вот почему важно понимать, что
представляют собой эти различные социальные медиа-платформы, и что
можно делать с ними в реалиях бизнеса. 

На сегодняшний день все больший оборот набирают тематические социальные сети. Тематические соцаильные сети "--- это специализированные сообщества, созданные вокруг общего интереса, местоположения, потребности, события или принадлежности к чему-либо~\cite{verticalSN}. 
Facebook, LinkedIn доминирует в пространстве социальных сетей, однако есть огромное количество людей, которые ищут более подходящие сети, т.к. контент, который добавляют друзья, не удовлетворяет их интересам.
Они хотят общаться с единомышленниками, которые ориентированы на конкретные профессии (недвижимость, право) или интересы (спорт, техника). 

Целью данного дипломного проекта является разработка программного средства 
для упрощения коммуникации больших групп пользователей,
интересующихся определённой предметной областью, в данном случае -
мотоциклами. Оно позволит людям общаться между собой, находить новые
знакомства, искать интересующую информацию, вести свой личный блог.