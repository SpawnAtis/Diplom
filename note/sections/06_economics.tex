\newcommand{\byn}{\text{руб.}}
\newcommand{\manhour}{\text{чел./ч.}}

\newcommand{\totalloc}{\text{V}_\text{о}}
\newcommand{\normativelaboriousness}{\text{Т}_\text{н}}
\newcommand{\complexity}{\text{К}_\text{с}}
\newcommand{\stdmodules}{\text{К}_\text{т}}
\newcommand{\novelty}{\text{К}_\text{н}}
\newcommand{\totallaboriousness}{\text{Т}_\text{о}}

\newcommand{\effectivetimefund}{\text{Ф}_\text{эф}}
\newcommand{\daysinyear}{\text{Д}_\text{г}}
\newcommand{\holidays}{\text{Д}_\text{п}}
\newcommand{\weekends}{\text{Д}_\text{в}}
\newcommand{\vacationdays}{\text{Д}_\text{о}}

\newcommand{\developersnumber}{\text{Ч}_\text{р}}
\newcommand{\developmenttime}{\text{Т}_\text{р}}
\newcommand{\developertimefundsymbol}{\text{Ф}_\text{пi}}

\newcommand{\firstratetariffsymbol}{\text{Т}_\text{ч}^1}
\newcommand{\averagehourspermonthsymbol}{\text{Ф}_\text{р}}
\newcommand{\hourspershiftsymbol}{\text{Т}_\text{ч}}
\newcommand{\bonusratesymbol}{K}

\newcommand{\basewagesymbol}{\text{З}_\text{о}}
\newcommand{\additionalwageratesymbol}{\text{Н}_\text{д}}
\newcommand{\additionalwagesymbol}{\text{З}_\text{д}}

\newcommand{\onecopyprofit}{\text{П}_\text{ед}}

\newcommand{\ssfratesymbol}{\text{Н}_\text{сз}}
\newcommand{\ssfchargessymbol}{\text{З}_\text{сз}}
\newcommand{\insuranceratesymbol}{\text{Н}_\text{ос}}
\newcommand{\insurancechargessymbol}{\text{З}_\text{ос}}

\newcommand{\consumablesratesymbol}{\text{Н}_\text{мз}}
\newcommand{\consumableschargessymbol}{\text{М}}

\newcommand{\machinetimeratesymbol}{\text{Н}_\text{мв}}
\newcommand{\machinehourpricesymbol}{\text{Ц}_\text{м}}
\newcommand{\machinetimechargessymbol}{\text{Р}_\text{м}}

\newcommand{\businesstripratesymbol}{\text{Н}_\text{рнк}}
\newcommand{\businesstripchargessymbol}{\text{Р}_\text{нк}}

\newcommand{\otherchargesratesymbol}{\text{Н}_\text{пз}}
\newcommand{\otherchargessymbol}{\text{П}_\text{з}}

\newcommand{\overheadratesymbol}{\text{Н}_\text{нр}}
\newcommand{\overheadchargessymbol}{\text{Р}_\text{н}}

\newcommand{\totalchargessymbol}{\text{С}_\text{п}}

\newcommand{\profitabilityratesymbol}{\text{У}_\text{рп}}
\newcommand{\profitabilitysymbol}{\text{П}_\text{о}}
\newcommand{\netprofitabilitysymbol}{\text{П}_\text{ч}}

\newcommand{\netpricesymbol}{\text{Ц}_\text{п}}

\newcommand{\vatsymbol}{\text{НДС}}
\newcommand{\vatsymbola}{\text{НДС}}
\newcommand{\vatratesymbol}{\text{Н}_\text{ДС}}

\newcommand{\sellingpricesymbol}{\text{Ц}_\text{о}}

\newcommand{\deploymentchargessymbol}{\text{Р}_\text{о}}
\newcommand{\deploymentratesymbol}{\text{Н}_\text{о}}
\newcommand{\maintenancechargessymbol}{\text{Р}_\text{с}}
\newcommand{\maintenanceratesymbol}{\text{Н}_\text{с}}

\newcommand{\capitalinvestmentsymbol}{\text{К}_\text{о}}
\newcommand{\purchasecostsymbol}{\text{К}_\text{пр}}
\newcommand{\deploymentcostsymbol}{\text{К}_\text{ос}}
\newcommand{\maintenancecostsymbol}{\text{К}_\text{с}}
\newcommand{\equipmentcostsymbol}{\text{К}_\text{тс}}
\newcommand{\assetscostsymbol}{\text{К}_\text{об}}

\newcommand{\averagewagesymbol}{\text{З}_\text{см}}
\newcommand{\baselabourisnesssymbol}{\text{Т}_\text{с1}}
\newcommand{\newlabourisnesssymbol}{\text{Т}_\text{с2}}
\newcommand{\basetaskscountsymbol}{\text{A}_\text{1}}
\newcommand{\newtaskscountsymbol}{\text{A}_\text{2}}

\newcommand{\wageeconomypertasksymbol}{\text{С}_\text{зе}}
\newcommand{\wageeconomysymbol}{\text{С}_\text{з}}
\newcommand{\totalwageeconomysymbol}{\text{С}_\text{н}}

\newcommand{\basedowntimesymbol}{\text{П}_\text{1}}
\newcommand{\newdowntimesymbol}{\text{П}_\text{2}}
\newcommand{\downtimepricesymbol}{\text{С}_\text{п}}
\newcommand{\downtimechargessymbol}{\text{С}_\text{с}}
\newcommand{\planserviceworktimesymbol}{\text{Д}_\text{рг}}

\newcommand{\totaleconomysymbol}{\text{С}_\text{о}}

\newcommand{\usernetprofitsymbol}{\text{П}_\text{ч}}
\newcommand{\profittaxratesymbol}{\text{Н}_\text{п}}

\newcommand{\countcopyssymbol}{\text{N}}

\newcommand{\sumprofitperyearsymbol}{\text{П}}

\newcommand{\clearProfitSymol}{\text{ЧП}}
\newcommand{\incomTaxRateSymbol}{\text{Н}_\text{п}}
\newcommand{\discontKSymbol}{\text{\(\alpha\)}_\text{i}}
\FPeval{\totalProgramSize}{17944}

\FPeval{\normativeManDays}{520}

\FPeval{\additionalComplexity}{clip(0.06+0.07)}

\FPeval{\stdModuleUsageFactor}{0.7}
\FPeval{\noveltyFactor}{0.9}

\FPeval{\daysInYear}{365}
\FPeval{\redLettersDaysInYear}{9}
\FPeval{\weekendDaysInYear}{103}
\FPeval{\vacationDaysInYear}{21}

\FPeval{\firstratetariff}{150}
\FPeval{\averagehourspermonth}{167.3}
\FPeval{\hourspershift}{8}
\FPeval{\bonusrate}{1.3}
\FPeval{\additionalwagerate}{20}
\FPeval{\ssfrate}{34}
\FPeval{\insurancerate}{0.6}
\FPeval{\consumablesrate}{5}
\FPeval{\machinetimerate}{15}
\FPeval{\machinetimereductionrate}{0.5}
\FPeval{\machinehourprice}{0.8}
\FPeval{\businesstriprate}{15}
\FPeval{\otherchargesrate}{18}
\FPeval{\overheadrate}{45}
\FPeval{\oneCopyPrice}{7}

\FPeval{\countfirstyearcopys}{1000}
\FPeval{\countsecondyearcopys}{1500}
\FPeval{\countthirdyearcopys}{2100}
\FPeval{\counfoursyearcopys}{3000}
\FPeval{\incomTaxRate}{0.18}

\FPeval{\profitabilityrate}{10}
\FPeval{\vatrate}{20}

\FPeval{\deploymentrate}{10}
\FPeval{\maintenancerate}{20}

\FPeval{\averagewage}{716.5}
\FPeval{\baselabourisness}{2}
\FPeval{\newlabourisness}{0.1}
\FPeval{\usesperday}{4}

\FPeval{\basedowntime}{50}
\FPeval{\newdowntime}{10}
\FPeval{\downtimeprice}{79.8}
\FPeval{\planserviceworktime}{300}

\FPeval{\profittaxrate}{18}
\FPeval{\refinancingTax}{0.1}



\FPeval{\totalProgramSizeCorrected}{\totalProgramSize}
\newcommand{\totallocfactor}{\num{\totalProgramSizeCorrected}}

\newcommand{\additionalcomplexityfactor}{\num{\additionalComplexity}}
\FPeval{\complexityFactor}{clip(1 + \additionalComplexity)}
\newcommand{\complexityfactor}{\num{\complexityFactor}}

\newcommand{\stdmodulesfactor}{\num{\stdModuleUsageFactor}}
\newcommand{\noveltyfactor}{\num{\noveltyFactor}}

\FPeval{\adjustedManDaysExact}{clip( \normativeManDays * \complexityFactor * \stdModuleUsageFactor * \noveltyFactor )}
\FPround{\adjustedManDays}{\adjustedManDaysExact}{0}
\newcommand{\normativelaboriousnessfactor}{\num{\normativeManDays}}
\newcommand{\totallaboriousnessfactor}{\num{\adjustedManDays}}


\FPeval{\workingDaysInYear}{ clip( \daysInYear - \redLettersDaysInYear - \weekendDaysInYear - \vacationDaysInYear ) }
\newcommand{\daysinyearfactor}{\num{\daysInYear}}
\newcommand{\holidaysfactor}{\num{\redLettersDaysInYear}}
\newcommand{\weekendsfactor}{\num{\weekendDaysInYear}}
\newcommand{\vacationdaysfactor}{\num{\vacationDaysInYear}}
\newcommand{\effectivetimefundfactor}{\num{\workingDaysInYear}}

\FPeval{\requiredNumberOfProgrammers}{3}
\newcommand{\developersnumberfactor}{\num{\requiredNumberOfProgrammers}}
\FPeval{\developmentTimeYearsExact}{\adjustedManDays / (\requiredNumberOfProgrammers * \workingDaysInYear)}
\FPround{\developmentTimeYears}{\developmentTimeYearsExact}{2}
\newcommand{\developmenttimeyearsfactor}{\num{\developmentTimeYears}}
\FPeval{\developmentTimeMonthsExact}{\developmentTimeYearsExact * 12}
\FPround{\developmentTimeMonths}{\developmentTimeMonthsExact}{1}
\newcommand{\developmenttimemonthsfactor}{\num{\developmentTimeMonths}}
\FPeval{\developmentTimeDaysExact}{\developmentTimeYearsExact * \daysInYear}
\FPround{\developmentTimeDays}{\developmentTimeDaysExact}{0}
\newcommand{\developmenttimefactor}{\num{\developmentTimeDays}}

\FPeval{\developertimefund}{round(\developmentTimeDays / \requiredNumberOfProgrammers,0)}
\newcommand{\developertimefundvalue}{\num{\developertimefund}}

\FPeval{\employeeAMonthExact}{\firstratetariff * 3.98}
\FPeval{\employeeBMonthExact}{\firstratetariff * 3.48}
\FPeval{\employeeCMonthExact}{\firstratetariff * 2.65}
\FPeval{\employeeAHourExact}{\employeeAMonthExact / \averagehourspermonth}
\FPeval{\employeeBHourExact}{\employeeBMonthExact / \averagehourspermonth}
\FPeval{\employeeCHourExact}{\employeeCMonthExact / \averagehourspermonth}
\FPround{\employeeAMonth}{\employeeAMonthExact}{2}
\FPround{\employeeBMonth}{\employeeBMonthExact}{2}
\FPround{\employeeCMonth}{\employeeCMonthExact}{2}
\FPround{\employeeAHour}{\employeeAHourExact}{2}
\FPround{\employeeBHour}{\employeeBHourExact}{2}
\FPround{\employeeCHour}{\employeeCHourExact}{2}
\newcommand{\employeeamonthwage}{\num{\employeeAMonth}}
\newcommand{\employeebmonthwage}{\num{\employeeBMonth}}
\newcommand{\employeecmonthwage}{\num{\employeeCMonth}}
\newcommand{\employeeahourwage}{\num{\employeeAHour}}
\newcommand{\employeebhourwage}{\num{\employeeBHour}}
\newcommand{\employeechourwage}{\num{\employeeCHour}}

\newcommand{\firstratetariffvalue}{\num{\firstratetariff}}
\newcommand{\averagehourspermonthvalue}{\num{\averagehourspermonth}}
\newcommand{\hourspershiftvalue}{\num{\hourspershift}}
\newcommand{\bonusratevalue}{\num{\bonusrate}}

\FPeval{\basewageExact}{(\employeeBHour + \employeeCHour) * \hourspershift * \developertimefund * \bonusrate}
\FPround{\basewage}{\basewageExact}{2}
\newcommand{\basewagevalue}{\num{\basewage}}

\newcommand{\additionalwageratevalue}{\num{\additionalwagerate}\%}
\FPeval{\additionalwage}{round(\basewage * \additionalwagerate / 100, 2)}
\newcommand{\additionalwagevalue}{\num{\additionalwage}}

\FPeval{\ssfcharges}{round((\basewage + \additionalwage) * \ssfrate / 100, 2)}
\FPeval{\insurancecharges}{round((\basewage + \additionalwage) * \insurancerate / 100, 2)}
\newcommand{\ssfchargesvalue}{\num{\ssfcharges}}
\newcommand{\ssfratevalue}{\num{\ssfrate}\%}
\newcommand{\insurancechargesvalue}{\num{\insurancecharges}}
\newcommand{\insuranceratevalue}{\num{\insurancerate}\%}

\FPeval{\consumablescharges}{round(\basewage * \consumablesrate / 100, 2)}
\newcommand{\consumablesratevalue}{\num{\consumablesrate}\%}
\newcommand{\consumableschargesvalue}{\num{\consumablescharges}}

\FPeval{\machinetimecharges}{round(\machinehourprice * \totalProgramSize / 100 * \machinetimerate * \machinetimereductionrate, 2)}
\newcommand{\machinetimechargesvalue}{\num{\machinetimecharges}}
\newcommand{\machinetimeratevalue}{\num{\machinetimerate}\%}
\newcommand{\machinetimereductionratevalue}{\num{\machinetimereductionrate}}
\newcommand{\machinehourpricevalue}{\num{\machinehourprice}}

\FPeval{\businesstripcharges}{round(\basewage * \businesstriprate / 100, 2)}
\newcommand{\businesstripratevalue}{\num{\businesstriprate}\%}
\newcommand{\businesstripchargesvalue}{\num{\businesstripcharges}}

\FPeval{\othercharges}{round(\basewage * \otherchargesrate / 100, 2)}
\newcommand{\otherchargesratevalue}{\num{\otherchargesrate}\%}
\newcommand{\otherchargesvalue}{\num{\othercharges}}

\FPeval{\overheadcharges}{round(\basewage * \overheadrate / 100, 2)}
\newcommand{\overheadratevalue}{\num{\overheadrate}\%}
\newcommand{\overheadvalue}{\num{\overheadcharges}}

\FPeval{\totalcharges}{\basewage + \additionalwage + \ssfcharges + \insurancecharges + \consumablescharges + \machinetimecharges + \businesstripcharges + \othercharges + \overheadcharges}
\FPeval{\totalchargesclipped}{clip(\totalcharges)}
\newcommand{\totalchargesvalue}{\num{\totalchargesclipped}}

\FPeval{\profitability}{round(\totalcharges * \profitabilityrate / 100, 2)}
\newcommand{\profitabilityratevalue}{\num{\profitabilityrate}\%}
\newcommand{\profitabilityvalue}{\num{\profitability}}

\FPeval{\netprice}{clip(\totalcharges + \profitability)}
\newcommand{\netpricevalue}{\num{\netprice}}

\FPeval{\vat}{round(\netprice * \vatrate / 100, 2)}
\newcommand{\vatvalue}{\num{\vat}}
\newcommand{\vatratevalue}{\num{\vatrate}\%}

\FPeval{\sellingprice}{clip(\netprice + \vat)}
\newcommand{\sellingpricevalue}{\num{\sellingprice}}

\FPeval{\vatsymbolaeval}{round((\oneCopyPrice * \vatrate) / (100 + \vatrate), 2)}
\newcommand{\vatsymbolavalue}{\num{\vatsymbolaeval}}

\FPeval{\countcopys}{clip(\countfirstyearcopys + \countsecondyearcopys + \countthirdyearcopys + \counfoursyearcopys)}

\FPeval{\onecopyprofiteval}{round(\oneCopyPrice - \vatsymbolaeval - (\sellingprice / \countcopys), 2)}
\newcommand{\onecopyprofitvalue}{\num{\onecopyprofiteval}}
%-----------------
\FPeval{\profitperyearvalueoneeval}{clip(\onecopyprofiteval * \countfirstyearcopys)}
\newcommand{\profitperyearvalueone}{\num{\profitperyearvalueoneeval}}
\FPeval{\profitperyearvaluetwoeval}{clip(\onecopyprofiteval * \countsecondyearcopys)}
\newcommand{\profitperyearvaluetwo}{\num{\profitperyearvaluetwoeval}}
\FPeval{\profitperyearvaluethreeeval}{clip(\onecopyprofiteval * \countthirdyearcopys)}
\newcommand{\profitperyearvaluethree}{\num{\profitperyearvaluethreeeval}}
\FPeval{\profitperyearvaluefoureval}{clip(\onecopyprofiteval * \counfoursyearcopys)}
\newcommand{\profitperyearvaluefour}{\num{\profitperyearvaluefoureval}}
%-----------------
%-----------------
\FPeval{\clearProfitValueOneEval}{clip(\profitperyearvalueoneeval - \profitperyearvalueoneeval * \incomTaxRate)}
\newcommand{\clearProfitValueOne}{\num{\clearProfitValueOneEval}}
\FPeval{\clearProfitValueTwoEval}{clip(\profitperyearvaluetwoeval - \profitperyearvaluetwoeval * \incomTaxRate)}
\newcommand{\clearProfitValueTwo}{\num{\clearProfitValueTwoEval}}
\FPeval{\clearProfitValueThreeEval}{clip(\profitperyearvaluethreeeval - \profitperyearvaluethreeeval * \incomTaxRate)}
\newcommand{\clearProfitValueThree}{\num{\clearProfitValueThreeEval}}
\FPeval{\clearProfitValueFourEval}{clip(\profitperyearvaluefoureval - \profitperyearvaluefoureval * \incomTaxRate)}
\newcommand{\clearProfitValueFour}{\num{\clearProfitValueFourEval}}
%-----------------
%-----------------
\FPeval{\discontKValueOneEval}{round(pow(-1, 1.1), 2)}
\newcommand{\discontKValueOne}{\num{\discontKValueOneEval}}
\FPeval{\discontKValueTwoEval}{round(pow(-2, 1.1), 2)}
\newcommand{\discontKValueTwo}{\num{\discontKValueTwoEval}}
\FPeval{\discontKValueThreeEval}{round(pow(-3, 1.1), 2)}
\newcommand{\discontKValueThree}{\num{\discontKValueThreeEval}}
%-----------------

\FPeval{\deploymentcharges}{round(\totalcharges * \deploymentrate / 100, 2)}
\newcommand{\deploymentratevalue}{\num{\deploymentrate}\%}
\newcommand{\deploymentchargesvalue}{\num{\deploymentcharges}}

\FPeval{\maintenancecharges}{round(\totalcharges * \maintenancerate / 100, 2)}
\newcommand{\maintenanceratevalue}{\num{\maintenancerate}\%}
\newcommand{\maintenancechargesvalue}{\num{\maintenancecharges}}

\FPeval{\netprofitability}{round(\profitability * (100 - \profittaxrate) / 100, 2)}
\newcommand{\netprofitabilityvalue}{\num{\netprofitability}}

\FPeval{\capitalinvestment}{clip(\maintenancecharges + \deploymentcharges + \sellingprice)}
\newcommand{\capitalinvestmentvalue}{\num{\capitalinvestment}}

\FPeval{\wageeconomypertask}{round(\averagewage * (\baselabourisness - \newlabourisness) / \averagehourspermonth, 2)}
\newcommand{\wageeconomypertaskvalue}{\num{\wageeconomypertask}}
\newcommand{\averagewagevalue}{\num{\averagewage}}
\newcommand{\baselabourisnessvalue}{\num{\baselabourisness}}
\newcommand{\newlabourisnessvalue}{\num{\newlabourisness}}

\FPeval{\taskscount}{clip(\usesperday * \daysInYear)}
\newcommand{\taskscountvalue}{\num{\taskscount}}
\newcommand{\usesperdayvalue}{\num{\usesperday}}

\FPeval{\wageeconomy}{clip(\wageeconomypertask * \taskscount)}
\newcommand{\wageeconomyvalue}{\num{\wageeconomy}}

\FPeval{\totalwageeconomy}{round(\wageeconomy * (100 + \bonusrate) / 100,2)}
\newcommand{\totalwageeconomyvalue}{\num{\totalwageeconomy}}

\FPeval{\downtimecharges}{round((\basedowntime - \newdowntime) * \planserviceworktime * \downtimeprice / 60,2)}
\newcommand{\downtimechargesvalue}{\num{\downtimecharges}}
\newcommand{\basedowntimevalue}{\num{\basedowntime}}
\newcommand{\newdowntimevalue}{\num{\newdowntime}}
\newcommand{\downtimepricevalue}{\num{\downtimeprice}}
\newcommand{\planserviceworktimevalue}{\num{\planserviceworktime}}

\FPeval{\totaleconomy}{clip(\totalwageeconomy + \downtimecharges)}
\newcommand{\totaleconomyvalue}{\num{\totaleconomy}}

\FPeval{\usernetprofit}{round(\totaleconomy * (1 - \profittaxrate / 100),2)}
\newcommand{\profittaxratevalue}{\num{\profittaxrate}\%}
\newcommand{\usernetprofitvalue}{\num{\usernetprofit}}

\FPeval{\usernetprofityearone}{round(\usernetprofit * 0.8696,2)}
\newcommand{\usernetprofityearonevalue}{\num{\usernetprofityearone}}
\FPeval{\usernetprofityeartwo}{round(\usernetprofit * 0.7561,2)}
\newcommand{\usernetprofityeartwovalue}{\num{\usernetprofityeartwo}}
\FPeval{\usernetprofityearthree}{round(\usernetprofit * 0.6575,2)}
\newcommand{\usernetprofityearthreevalue}{\num{\usernetprofityearthree}}

\FPeval{\excessovercostsyearzero}{clip(-\capitalinvestment)}
\newcommand{\excessovercostsyearzerovalue}{\num{\excessovercostsyearzero}}
\FPeval{\excessovercostsyearone}{clip(\usernetprofityearone)}
\newcommand{\excessovercostsyearonevalue}{\num{\excessovercostsyearone}}
\FPeval{\excessovercostsyeartwo}{clip(\usernetprofityeartwo)}
\newcommand{\excessovercostsyeartwovalue}{\num{\excessovercostsyeartwo}}
\FPeval{\excessovercostsyearthree}{clip(\usernetprofityearthree)}
\newcommand{\excessovercostsyearthreevalue}{\num{\excessovercostsyearthree}}

\FPeval{\excessovercostswithtimingyearone}{clip(\excessovercostsyearzero + \usernetprofityearone)}
\newcommand{\excessovercostswithtimingyearonevalue}{\num{\excessovercostswithtimingyearone}}
\FPeval{\excessovercostswithtimingyeartwo}{clip(\excessovercostswithtimingyearone + \usernetprofityeartwo)}
\newcommand{\excessovercostswithtimingyeartwovalue}{\num{\excessovercostswithtimingyeartwo}}
\FPeval{\excessovercostswithtimingyearthree}{clip(\excessovercostswithtimingyeartwo + \usernetprofityearthree)}
\newcommand{\excessovercostswithtimingyearthreevalue}{\num{\excessovercostswithtimingyearthree}}


\section{Технико-экономическое обоснование эффективности разработки и реализации программного средства}
\label{sec:economics}

\subsection{Характеристика программного средства}
\label{sec:economics:description}

Существует большое количество социальных сетей, которые решают проблемы коммуникации, однако все больше людей хотят общаться с единомышленниками, которые ориентированы на конкретные профессии или интересы, приэтом получая от туда тематическией контент.
Вследствие этого появляется необходимость в создании тематических социальных сетей, которые бы комбинировали функционал обычных социальных сетей и тематико-ориентированный функционал.

Для рекламодателей социальные сети предоставляют уникальные
возможности непосредственного контакта с потребителями. Ежедневно
миллионы пользователей ведут беседы о компаниях, их товарах и услугах,
делясь своим мнением и впечатлениями. В результате отдельно взятый
участник сетевого сообщества может испортить (или наоборот улучшить)
репутацию компании с многомиллионным оборотом.

Это то, что разделяет платформы социальных сетей и делает их
полезными для разных случаев. Вот почему важно понимать, что
представляют собой эти различные социальные медиа-платформы, и что
можно делать с ними в реалиях бизнеса. 

На сегодняшний день все больший оборот набирают тематические социальные сети. Тематические соцаильные сети "--- это специализированные сообщества, созданные вокруг общего интереса, местоположения, потребности, события или принадлежности к чему-либо~\cite{verticalSN}. 
Facebook, LinkedIn доминирует в пространстве социальных сетей, однако есть огромное количество людей, которые ищут более подходящие сети, т.к. контент, который добавляют друзья, не удовлетворяет их интересам.
Они хотят общаться с единомышленниками, которые ориентированы на конкретные профессии (недвижимость, право) или интересы (спорт, техника). 

Целью данного дипломного проекта является позволит создать комфортные условия для построения взаимоотношений пользователей и их коммуникации в социальной сети (на примере сообщества мотоциклистов), 
путем создания кроссплатформенного мобильного приложения.

Данное приложение решает вышеперечисленные проблемы, путем объединения функционала мобильного навигатора и социальной сети, как Instagram.

Исходя из текущей тенденций на рынке мобильных приложений и маркетинвого исследования, приложение будет восстребовано на рынке в течение четырех лет:
в 2019 году планируется реализовать $ \countfirstyearcopys $ копий, в 2020 году - $ \countsecondyearcopys $ копий, в 2021 г. - $ \countthirdyearcopys $ копий и $ \counfoursyearcopys $ копий в 2020 году.

Экономическая целесообразность инвестиций в разработку и использование мобильного приложения осуществляется на основе расчета следующих показателей~\cite{nosenko}:
\begin{itemize}
	\item чистая дисконтированная стоимость ЧДД;
	\item срок окупаемости инвестиций ($ \text{T}_\text{ок} $));
	\item рентабельность инвестиций ($ \text{Р}_\text{н} $);
\end{itemize}

\subsection{Расчет сметы затрат на разработку и отпускной цены}
\label{sec:economics:estimate}

%Основной статьей расходов на создание ПО является заработная плата разработчиков проекта. Информация об исполнителях перечислена в таблице~\ref{table:economics:estimate:employees}. Кроме того, в таблице приведены данные об их тарифных разрядах, приведены разрядные коэффициенты, а также по формулам~\ref{eq:economics:estimate:month_wage} и~\ref{eq:economics:estimate:hour_wage} рассчитаны месячный и часовой оклады.
Затраты на основную ЗП (заработную плату) команды разработчиков определяются исходя из состава и численности команды, размеров месячной заработной платы каждого из участников команды, а также общей трудоемкости разработки программного обеспечения. Кроссплатформенная разработка существенное сократила количество разработчиков, что повлекло к снижению затрат на его разработку. Расчет велечины основной заработной платы участников команды осуществляется по формуле:

\begin{equation}
	\begin{aligned}
		\basewagesymbol &= \sum_{i=1}^n \text{Т}_\text{чi} \cdot \text{Т}_\text{ч} \cdot \developertimefundsymbol \cdot K,
	\end{aligned}
	\end{equation}
	\bigbreak
	\begin{explanation}
	где & $ \text{n} $ & количество исполнителей;\\
		& $ \text{Т}_\text{чi} $ & часовая тарифная ставка i-го исполнителя, \byn;\\
		& $ \text{Т}_\text{ч} $ & количество часов работы в день;\\
		& $ \text{Ф}_\text{пi} $ & плановый фонд рабочего времени i-го исполнителя, д.;\\
		& $ K $ & коэффициент премирования (принятый равным \bonusratevalue).
	\end{explanation}

	Месячный и часовой оклады рассчитываются по формулам~\ref{eq:economics:estimate:month_wage} и~\ref{eq:economics:estimate:hour_wage}.

	\begin{equation}
		\label{eq:economics:estimate:month_wage}
			\text{T}_\text{м} = \text{T}_\text{м}^1 \cdot \text{T}_\text{к}, 
		\end{equation}
		\bigbreak
		\begin{equation}
		\label{eq:economics:estimate:hour_wage}
			\text{T}_\text{ч} = \frac{\text{T}_\text{м}}{\text{Ф}_\text{р}},
		\end{equation}
		\bigbreak
		\begin{explanation}
		где & $ \text{T}_\text{м} $ & месячный оклад;\\
			& $ \text{T}_\text{м}^1 $ & тарифная ставка 1-го разряда (положим ее равной \num{\firstratetariff} \byn);\\
			& $ \text{T}_\text{к} $ & тарифный коэффициент;\\
			& $ \text{T}_\text{ч} $ & часовой оклад;\\
			& $ \text{Ф}_\text{р} $ & среднемесячная норма рабочего времени (в 2019 г. составляет \num{167.3} ч.~\cite{labour_calendar}).
		\end{explanation}

\begin{table}[H]
  \caption{Расчет основной заработной платы}
  \label{table:economics:estimate:employees}
  \begin{tabular}{| >{\raggedright}m{0.174\textwidth} 
                  | >{\centering}m{0.086\textwidth} 
                  | >{\centering}m{0.132\textwidth} 
                  | >{\centering}m{0.128\textwidth} 
									| >{\centering}m{0.1\textwidth} 
                  | >{\centering}m{0.128\textwidth} 
									| >{\centering\arraybackslash}m{0.10\textwidth}|}
	\hline
	Исполнители & Разряд & Тарифный коэффициент & Месячный оклад, \byn & Часовой оклад, \byn & Плановый фонд рабочего времени & ЗП \\

	\hline
	Руководитель проекта & 18 & \num{4.26} & \employeeamonthwage & \employeeahourwage & \countOfHoursPerDayA & \basewagevalueA \\

	\hline
	Ведущий инженер-программист & 15 & \num{4.88} & \employeebmonthwage & \employeebhourwage & \countOfHoursPerDayB & \basewagevalueB \\

	\hline
	Инженер-программист & 10 & \num{2.48} & \employeecmonthwage & \employeechourwage & \countOfHoursPerDayC & \basewagevalueC \\
	\hline
	Итого & & & & & & \zpItogvalue \\
	\hline
	Премия & & & & & & \num{\premia} \\
	\hline
	Основная заработная плата & & & & & & \basewagevalue \\
	\hline
  \end{tabular}
\end{table}

Дополнительные выплаты рассчитваются по формуле:\\
\begin{equation}
	\additionalwagesymbol = \frac{\basewagesymbol \cdot \additionalwageratesymbol}{100\%},
	\end{equation}
	\bigbreak
	\begin{explanation}
	где & $ \additionalwageratesymbol $ &норматив дополнительной заработной платы ($\additionalwageratevalue$).
	\end{explanation}

Тогда найдем дополнительные выплаты:
\begin{flalign*}
	\hspace*{6ex} 
	\text{З}_\text{д} = \frac{\basewagevalue \cdot \additionalwageratevalue}{100\%} = \additionalwagevalue~\byn&&&
	\end{flalign*}
\bigbreak

Отчисления в фонд социальной защиты населения и в фонд обязательного страхования определяются в соответствии с действующим законодательством по нормативу в процентном отношении к фонду основной и дополнительной зарплат по следующим формулам
\vspace{-0.8mm}
\begin{equation}
	\ssfchargessymbol = \frac{(\basewagesymbol + \additionalwagesymbol) \cdot \ssfratesymbol}{100\%}, 
	\end{equation}
	\bigbreak
	\begin{equation}
		\insurancechargessymbol = \frac{(\basewagesymbol + \additionalwagesymbol) \cdot \insuranceratesymbol}{100\%}.
	\end{equation}
	\bigbreak
\vspace{1em}

В настоящее время нормы отчислений в ФСЗН $\ssfratesymbol~=~\ssfratevalue$ и в фонд обязательного страхования $\insuranceratesymbol~=~\insuranceratevalue$. Исходя из этого, размеры отчислений
\begin{flalign*}
		\hspace*{6ex} 
	\ssfchargessymbol = \frac{(\basewagevalue + \additionalwagevalue) \cdot \ssfratevalue}{100\%} = \ssfchargesvalue~\byn,&&&
\end{flalign*}
\begin{flalign*}
	\hspace*{6ex} 
\insurancechargessymbol = \frac{(\basewagevalue + \additionalwagevalue) \cdot \insuranceratevalue}{100\%} = \insurancechargesvalue~\byn&&&
\end{flalign*}

Расходы по статье <<Машинное время>> включают оплату машинного времени, необходимого для разработки и отладки ПО, которое определяется по нормативам на 100 строк исходного кода. 

Норматив зависит от характера решаемых задач и типа ПК; для текущего проекта примем $\machinetimeratesymbol~=~\machinetimeratevalue$~\cite[приложение 6]{palitsyn}.

\begin{equation}
	\machinetimechargessymbol = \machinehourpricesymbol \cdot \frac{\totalloc}{100} \cdot \machinetimeratesymbol,
\end{equation}
\begin{explanation}
	где & $ \machinehourpricesymbol $ & велечина стоимости машинного часа;\\
		& $ \totalloc $ & количество строк исходного кода;\\
		& $ \machinetimeratesymbol $ & норматив машинного времени.
	\end{explanation}

Примем величину стоимости машино-часа $\machinehourpricesymbol~=~\machinehourpricevalue~\byn$

Тогда, применяя понижающий коэффициент \machinetimereductionratevalue, получим величину расходов
\begin{flalign*}
	\hspace*{6ex} 
	\machinetimechargessymbol = \machinehourpricevalue \cdot \frac{\totallocfactor}{100} \cdot \machinetimeratevalue \cdot \machinetimereductionratevalue = \machinetimechargesvalue~\byn&&&
\end{flalign*}

Расходы по статье <<Научные командировки>> определяются по нормативу в процентах к основной заработной плате. 

\begin{equation}
	\businesstripchargessymbol = \frac{\basewagesymbol \cdot \businesstripratesymbol}{100\%},
\end{equation}
\begin{explanation}
	где & $ \businesstripratesymbol $ & норматив начных командировок.
	\end{explanation}
\vspace{0.3ex}

Принимая норматив равным $\businesstripratesymbol~=~\businesstripratevalue$ получим величину расходов
\begin{flalign*}
	\hspace*{6ex} 
	\businesstripchargessymbol = \frac{\basewagevalue \cdot \businesstripratevalue}{100\%} = \businesstripchargesvalue~\byn&&&
\end{flalign*}

Расходы по статье <<Прочие затраты>> включают затраты на приобретение и подготовку специальной научно-технической информации и специальной литературы.

Определяются по нормативу в процентах к основной заработной плате. 
\begin{equation}
	\otherchargessymbol = \frac{\basewagesymbol \cdot \otherchargesratesymbol}{100\%},
\end{equation}
\begin{explanation}
	где & $\otherchargesratesymbol $ & норматив прочих затрат.
	\end{explanation}
\vspace{0.3ex}

Принимая норматив равным $\otherchargesratesymbol~=~\otherchargesratevalue$ получим величину расходов
\begin{flalign*}
	\hspace*{6ex} 
	\otherchargessymbol = \frac{\basewagevalue \cdot \otherchargesratevalue}{100\%} = \otherchargesvalue~\byn&&&
\end{flalign*}

Затраты по статье <<Накладные расходы>>, связанные с необходимостью  содержания  аппарата  управления,  вспомогательных хозяйств и опытных (экспериментальных) производств.

Также к ним относятся расходы на общехозяйственные нужды, относятся к конкретному ПО по нормативу в процентном отношении к основной заработной плате
исполнителей. 
\begin{equation}
	\overheadchargessymbol = \frac{\basewagesymbol \cdot \overheadratesymbol}{100\%},
\end{equation}
\begin{explanation}
	где & $ \overheadratesymbol $ & норматив наклодных расходов.
	\end{explanation}

Принимая норматив равным $\overheadratesymbol~=~\overheadratevalue$ получим величину расходов
\begin{flalign*}
	\hspace*{6ex} 
	\overheadchargessymbol = \frac{\basewagevalue \cdot \otherchargesratevalue}{100\%} = \otherchargesvalue~\byn&&&
\end{flalign*}

Общая сумма расходов по смете определяется как сумма вышерассчитанных показателей

\begin{equation}
\begin{aligned}
	\totalchargessymbol = \basewagesymbol + \additionalwagesymbol + \ssfchargessymbol &+ \insurancechargessymbol + \machinetimechargessymbol + \businesstripchargessymbol + \otherchargessymbol + \overheadchargessymbol =\\
	&= \totalchargesvalue~\byn
\end{aligned}
\end{equation}

\begin{equation}
	\profitabilitysymbol = \frac{\totalchargessymbol \cdot \profitabilityratesymbol}{100\%},
\end{equation}
\bigbreak
\begin{explanation}
	где & $ \profitabilityratesymbol $ & норматив наклодных расходов.
	\end{explanation}

	Рентабельность определяется из результатов анализа рыночных условий и переговоров с потребителями ПО.

	Приняв уровень рентабельности $\profitabilityratesymbol~=~\profitabilityratevalue$, найдем прибыль от реализации программного обепечения
\begin{flalign*}
	\hspace*{6ex} 
	\profitabilitysymbol = \frac{\totalchargesvalue \cdot \profitabilityratevalue}{100\%} = \profitabilityvalue~\byn&&&
\end{flalign*}

Т.к. IT-компания является резидентом ПВТ, согласно законодательства РБ, такое предприятие освобождается от уплаты налога на прибыль и налога на добавленную стоимость.

На основании расчета прибыли и уровня себестоимости рассчитаем прогнозируемую цену

\begin{equation}
	\netpricesymbol = \totalchargessymbol + \profitabilitysymbol = \totalchargesvalue + \profitabilityvalue = \netpricevalue~\byn
\end{equation}

\subsection{Оценка экономического эффекта от продажи программного средства}
\label{sec:economics:effect}

Положительный экономический эффект для разработчика программного средства достигается за счет получения прибыли от реализация мобильного приложения большому количеству пользователей. Прибыль достигаетсяза счет реализации мобильного приложения на рынку мобильных приложений. Прибыль от реализации зависит от количества клиентов, ценны мобильного приложения и затрат на разработку.

Исходя из тенденций на рынке мобильных приложений и маркетингового исследования, приложение будет востребовано ны рынке в течение четырех лет: в 2019 году планируется реализовать \countfirstyearcopys~копий, в 2020 году --- \countsecondyearcopys~копий, 2021 год --- \countthirdyearcopys~копий и \counfoursyearcopys~в 2022 году.

На основании анализа цен на аналогичные приложения мы определим цену мобильного приложения: \oneCopyPrice~\byn

Т.к. резиденты ПВТ освобождаются от уплаты налога на прибыль и налога на добавленную стоимость, то чистая прибыль равна суммарной годовой прибыли.

Прибыль от продажи одной копии рассчитывается по формуле:
\begin{equation}
	\label{eq:economics:estimate:onecopyprofit}
	\begin{aligned}
		\onecopyprofit &= \text{Ц} - \frac{\text{З}_\text{р}}{\countcopyssymbol},
	\end{aligned}
	\end{equation}
	\bigbreak
	\begin{explanation}
	где & $ \text{Ц} $ & цена реализации одной копии, \byn;\\
		& $ \text{З}_\text{р} $ & сумма расходов на разработку и реализацию, \byn;\\
		& $ \countcopyssymbol $ & количество реализованных копий.
	\end{explanation}

 Подставим значения в формулу ~\ref{eq:economics:estimate:onecopyprofit}:
 \begin{flalign*}
	\hspace*{6ex} 
	\onecopyprofit = \oneCopyPrice - \frac{\sellingpricevalue}{\countcopys} = \onecopyprofitvalue~\byn&&&
\end{flalign*}

Cуммарная прибыль за год рассчитывется как:
\begin{equation}
	\sumprofitperyearsymbol = \text{П}_\text{ед} \cdot \countcopyssymbol,
\end{equation}

Следовательно прибыль от программного продукта по годам составляет:
\begin{flalign*}
	\hspace*{6ex} 
	\text{П}_\text{1} &= \onecopyprofitvalue \cdot \countfirstyearcopys = \profitperyearvalueone~\byn,\\
	\text{П}_\text{2} &= \onecopyprofitvalue \cdot \countsecondyearcopys = \profitperyearvaluetwo~\byn,\\
	\text{П}_\text{3} &= \onecopyprofitvalue \cdot \countthirdyearcopys = \profitperyearvaluethree~\byn,\\
	\text{П}_\text{4} &= \onecopyprofitvalue \cdot \counfoursyearcopys = \profitperyearvaluefour~\byn&&&
\end{flalign*}

\subsection{Расчет показателей эффективности}
\label{sec:economics:effect}

В ходе реализации ПС чистая прибыль в конечном итоге возмещает расходы на разработку.
Однако полученные при этом суммы результатов (прибыли) и затрат по годам приводят к единому времени --- расчетному году (за расчетный год принят 2019 год) путем умножения результатов затрат за каждый год на коэффициент дисконтирования $\discontKSymbol$, который рассчитывается по формуле:
\begin{equation}
	\begin{aligned}
		\discontKSymbol &= \text{1 + E}^{\text{t}_{\text{p}} - t},
	\end{aligned}
\end{equation}
\bigbreak
\begin{explanation}
	где & $ E $ & норматив приведения равновременных затрат и результатов;\\
		& $ \text{t}_{\text{p}} $ & номер года, результаты которого приводятся к расчетному;\\
		& $ t $ & номер расчетного года.
\end{explanation}

На 01.05.2019 ставка рефинансирования состовляет 10,5\%~\cite{refinancingTax}.
Итого коэффициент дисконтирования по годам (2019 - 2022):
\begin{flalign*}
	\hspace*{6ex} 
	\text{\(\alpha\)}_\text{1} &= 1,\\
	\text{\(\alpha\)}_\text{2} &= \text{(1 + \refinancingTax)}^{\text{1 - 2}} = \discontKValueOne,\\
	\text{\(\alpha\)}_\text{3} &= \text{(1 + \refinancingTax)}^{\text{1 - 3}} = \discontKValueTwo,\\
	\text{\(\alpha\)}_\text{4} &= \text{(1 + \refinancingTax)}^{\text{1 - 4}} = \discontKValueThree.&&&
\end{flalign*}

Cведем данные расчета вэкономического эффекта в таблицу~\ref{table:economics:estimate:economEffect}.
\begin{table}[!ht]
  \caption{Расчет экономического эффекта}
  \label{table:economics:estimate:economEffect}
  \begin{tabular}{| >{\centering}m{0.31\textwidth} 
                  | >{\centering}m{0.107\textwidth} 
                  | >{\centering}m{0.107\textwidth} 
                  | >{\centering}m{0.107\textwidth} 
									| >{\centering}m{0.107\textwidth} 
                  | >{\centering\arraybackslash}m{0.1\textwidth}|}
\hline
		\multirow{2}{*}{Показатели} & \multirow{2}{*}{Ед. изм.} & \multicolumn{4}{c|}{Расчетный период} \\ \cline{3-6}
		 & & 2019 & 2020 & 2021 & 2022 \\ 
\hline
\multicolumn{6}{|c|}{Результаты:} \\ \hline
Экономический эффект & руб. & \multicolumn{1}{c|}{\profitperyearvalueone} & \profitperyearvaluetwo & \profitperyearvaluethree & \profitperyearvaluefour \\ \hline
Коэффициент дисконтирования & руб. & \multicolumn{1}{c|}{1} & \discontKValueOne & \discontKValueTwo & \discontKValueThree \\ \hline
Дисконтированный результат & руб. & \multicolumn{1}{c|}{\profitperyearvalueone} & 6763,5 & 8627,22 & 11122,2 \\ \hline
Затраты на разработку программного средства & руб. & \multicolumn{1}{c|}{\totalchargesvalue} & - & - & - \\ \hline
Дисконтированные инвестиции & руб. & \multicolumn{1}{c|}{\totalchargesvalue} & - & - & \- \\ \hline
ЧДД по годам & руб. & \multicolumn{1}{c|}{-7090,38} & 6763,5 & 8627,22 & 11122,2 \\ \hline
ЧДД с нарастающим итогом & руб. & \multicolumn{1}{c|}{-7090,38} & 326,88 & 8954,1 & 20076,3 \\ \hline
	\end{tabular}
\end{table}

Так как чистый дисконтированный доход больше нуля, то проект эффективен, то есть инвестиции в разработку данного ПО экономически целесообразны.

Рассчитаем рентабельность инвестиций в разработку и внедрение программного продукта $\rentInvestSymbol$ по формуле:
\begin{equation}
	\label{eq:economics:estimate:rentInvest}
	\begin{aligned}
		\rentInvestSymbol &= \frac{\clearProfitPerYearSymbol}{\text{З}} \cdot 100,
	\end{aligned}
\end{equation}
\bigbreak
\begin{explanation}
	где & $ \clearProfitPerYearSymbol $ & среднегодовая величина чистой прибыли за расчетный период, \byn
\end{explanation}

$ \clearProfitPerYearSymbol $ определяется по формуле:
\begin{equation}
	\label{eq:economics:estimate:clearProfitPerYear}
	\begin{aligned}
		\clearProfitPerYearSymbol = \frac{\sum_{i=1}^n \cdot \text{П}_\text{чt}}{n},
	\end{aligned}
\end{equation}
\bigbreak
\begin{explanation}
	где & $ \text{П}_\text{чt} $ & чистая прибыль, полученная в году t, \byn
\end{explanation}

По формуле~\ref{eq:economics:estimate:clearProfitPerYear} среднегодовую величину чистой прибыли за расчетный период:
\begin{flalign*}
	\hspace*{6ex} 
	\clearProfitPerYearSymbol = \frac{\profitperyearvalueone + \profitperyearvaluetwo + \profitperyearvaluethree + \profitperyearvaluefour}{4} = \clearProfitPerYearValue~\byn&&&
\end{flalign*}

Подставив данное значение в формулу~\ref{eq:economics:estimate:rentInvest} получим рентабельность инвестиций в разработку и внедрение программного продукта.
\begin{flalign*}
	\hspace*{6ex} 
	\rentInvestSymbol = \frac{\clearProfitPerYearValue}{\totalchargesvalue} = 79~\%.&&&
\end{flalign*}

В результате технико-экономического обоснование эффективности разработки и реализации программного модуля взаимодействия с мультимедиа объектами были получены следующие значения показателей эффективности:
\begin{itemize}
	\item чистый дисконтированный доход за четыре года продаж программы составит 11757,22 руб.;
	\item затраты на разработку программного продукта окупятся на второй год его использования;
	\item рентабельность инвестиций составит $79$~\%.
\end{itemize}

Полученные результаты свидетельствуют об эффективности разработки проектируемого программного средства.
