\input{sections/06_economics_macros}
\input{sections/06_economics_calculations}

\section{Технико-экономическое обоснование эффективности разработки и реализации программного средства}
\label{sec:economics}

\subsection{Характеристика программного средства}
\label{sec:economics:description}

Программное средство, разрабатываемое в рамках дипломного проекта, предназначено для упрощения коммуникации больших групп пользователей, интересующихся определённой предметной областью, в данном случае --- мотоциклами.
Существует большое количество социальных сетей, которые решают проблемы коммуникации, однако все больше людей хотят общаться с единомышленниками, которые ориентированы на конкретные профессии или интересы, и получать тематическией контент.
Вследствие этого появляется необходимость в создании тематических социальных сетей, которые бы комбинировали функционал обычных социальных сетей и тематико-ориентированный функционал.
Данные проблемы в контексте сообщества людей, увлекающимися мототематикой, могут быть решены путем создания мобильного приложения, которое бы объединяло функционалы мобильного навигатора и социальной сети, как Instagram.

Исходя из текущей тенденций на рынке мобильных приложений и маркетинвого исследования, приложение будет восстребовано на рынке в течение четырех лет:
в 2019 году планируется реализовать $ \countfirstyearcopys $ копий, в 2020 году - $ \countsecondyearcopys $ копий, в 2021 г. - $ \countthirdyearcopys $ копий и $ \counfoursyearcopys $ копий в 2020 году.

Разработки проектов программных средств связана со значительными затратами ресурсов. В связи с этим создание и реализация каждого проекта программного обеспечения нуждается в соответствующем тех\-ни\-ко-эко\-но\-ми\-чес\-ком обосновании~\cite{palitsyn}, которое и описывается в данном разделе.

Экономическая целесообразность инвестиций в разработку и использование мобильного приложения осуществляется на основе следующих показателей~\cite{nosenko}:
\begin{itemize}
	\item чистая дисконтированная стоимость ЧДД;
	\item срок окупаемости инвестиций ($ \text{T}_\text{ок} $));
	\item рентабельность инвестиций ($ \text{Р}_\text{н} $);
\end{itemize}

\subsection{Расчет суммы затрат на разработку и отпускной цены программного средства}
\label{sec:economics:estimate}

Основной статьей расходов на создание ПО является заработная плата разработчиков проекта. Информация об исполнителях перечислена в таблице~\ref{table:economics:estimate:employees}. Кроме того, в таблице приведены данные об их тарифных разрядах, приведены разрядные коэффициенты, а также по формулам~\ref{eq:economics:estimate:month_wage} и~\ref{eq:economics:estimate:hour_wage} рассчитаны месячный и часовой оклады.

\begin{table}[!ht]
  \caption{Работники, занятые в проекте}
  \label{table:economics:estimate:employees}
  \begin{tabular}{| >{\raggedright}m{0.3\textwidth} 
                  | >{\centering}m{0.09\textwidth} 
                  | >{\centering}m{0.18\textwidth} 
                  | >{\centering}m{0.15\textwidth} 
                  | >{\centering\arraybackslash}m{0.15\textwidth}|}
	\hline
	{\begin{center}Исполнители\end{center}} & Разряд & Тарифный коэффициент & Месячный оклад, \byn & Часовой оклад, \byn \\

	% \hline
	% Руководитель проекта & 17 & \num{3.98} & \employeeamonthwage & \employeeahourwage \\

	\hline
	Ведущий инженер-программист & 15 & \num{3.48} & \employeebmonthwage & \employeebhourwage\\

	\hline
	Инженер-программист II категории & 11 & \num{2.65} & \employeecmonthwage & \employeechourwage\\
	\hline
  \end{tabular}
\end{table}
\vspace{-6mm}

\begin{equation}
\label{eq:economics:estimate:month_wage}
	\text{T}_\text{м} = \text{T}_\text{м}^1 \cdot \text{T}_\text{к},
\end{equation}
\begin{equation}
\label{eq:economics:estimate:hour_wage}
	\text{T}_\text{ч} = \frac{\text{T}_\text{м}}{\text{Ф}_\text{р}},
\end{equation}
\begin{explanation}
где & $ \text{T}_\text{м} $ & месячный оклад;\\
	& $ \text{T}_\text{м}^1 $ & тарифная ставка 1-го разряда (положим ее равной \num{\firstratetariff} \byn);\\
	& $ \text{T}_\text{к} $ & тарифный коэффициент;\\
	& $ \text{T}_\text{ч} $ & часовой оклад;\\
	& $ \text{Ф}_\text{р} $ & среднемесячная норма рабочего времени (в 2019 г. составляет \num{167.3} ч.~\cite{labour_calendar}).
\end{explanation}

Тогда основная заработная плата исполнителей составит
\begin{equation}
\begin{aligned}
	\basewagesymbol &= \sum_{i=1}^n \text{Т}_\text{чi} \cdot \text{Т}_\text{ч} \cdot \developertimefundsymbol \cdot K = \\
	&= (\employeebhourwage + \employeechourwage) \cdot \hourspershiftvalue \cdot \developertimefundvalue \cdot \bonusratevalue = \basewagevalue~\text{\byn},
\end{aligned}
\end{equation}
\begin{explanation}
где & $ \text{Т}_\text{чi} $ & часовая тарифная ставка i-го исполнителя, \byn;\\
	& $ \text{Т}_\text{ч} $ & количество часов работы в день;\\
	& $ \text{Ф}_\text{пi} $ & плановый фонд рабочего времени i-го исполнителя, д.;\\
	& $ K $ & коэффициент премирования (принятый равным \bonusratevalue).
\end{explanation}

Дополнительная заработная плата включает выплаты, предусмотренные законодательство о труде: оплата отпусков, льготных часов, времени  выполнения  государственных обязанностей и других выплат, не связанных с основной деятельностью исполнителей, и определяется по нормативу, установленному в организации, в процентах к основной заработной плате.
Приняв норматив дополнительной заработной платы $\additionalwageratesymbol = \additionalwageratevalue$, рассчитаем дополнительные выплаты
\begin{equation}
	\additionalwagesymbol = \frac{\basewagesymbol \cdot \additionalwageratesymbol}{100\%} = \frac{\basewagevalue \cdot \additionalwageratevalue}{100\%} = \additionalwagevalue~\byn
\end{equation}

Отчисления в фонд социальной защиты населения и в фонд обязательного страхования определяются в соответствии с действующим законодательством по нормативу в процентном отношении к фонду основной и дополнительной зарплат по следующим формулам
\begin{equation}
\begin{aligned}
	\ssfchargessymbol &= \frac{(\basewagesymbol + \additionalwagesymbol) \cdot \ssfratesymbol}{100\%},\\
	\insurancechargessymbol &= \frac{(\basewagesymbol + \additionalwagesymbol) \cdot \insuranceratesymbol}{100\%}.
\end{aligned}
\end{equation} 

В настоящее время нормы отчислений в ФСЗН $\ssfratesymbol~=~\ssfratevalue$ и в фонд обязательного страхования $\insuranceratesymbol~=~\insuranceratevalue$. Исходя из этого, размеры отчислений
\begin{equation}
\begin{aligned}
	\ssfchargessymbol &= \frac{(\basewagevalue + \additionalwagevalue) \cdot \ssfratevalue}{100\%} = \ssfchargesvalue~\byn,\\[5mm]
	\insurancechargessymbol &= \frac{(\basewagevalue + \additionalwagevalue) \cdot \insuranceratevalue}{100\%} = \insurancechargesvalue~\byn\\[5mm]
\end{aligned}
\end{equation}

Расходы по статье <<Материалы>> отражают траты на магнитные носители, бумагу, красящие материалы, необходимые для разработки ПО определяются по нормативу к фонду основной заработной платы разработчиков. Исходя из принятого норматива $\consumablesratesymbol~=~\consumablesratevalue$ определим величину расходов

\begin{equation}
	\consumableschargessymbol = \frac{\basewagesymbol \cdot \consumablesratesymbol}{100\%} = \frac{\basewagevalue \cdot \consumablesratevalue}{100\%} = \consumableschargesvalue~\byn\\[5mm]
\end{equation}

Расходы по статье <<Машинное время>> включают оплату машинного времени, необходимого для разработки и отладки ПО, которое определяется по нормативам на 100 строк исходного кода. Норматив зависит от характера решаемых задач и типа ПК; для текущего проекта примем $\machinetimeratesymbol~=~\machinetimeratevalue$~\cite[приложение 6]{palitsyn}. Примем величину стоимости машино-часа $\machinehourpricesymbol~=~\machinehourpricevalue~\byn$ Тогда, применяя понижающий коэффициент \machinetimereductionratevalue, получим величину расходов
\begin{equation}
	\machinetimechargessymbol = \machinehourpricesymbol \cdot \frac{\totalloc}{100} \cdot \machinetimeratesymbol = \machinehourpricevalue \cdot \frac{\totallocfactor}{100} \cdot \machinetimeratevalue \cdot \machinetimereductionratevalue = \machinetimechargesvalue~\byn
\end{equation}

Расходы по статье <<Научные командировки>> определяются по нормативу в процентах к основной заработной плате. Принимая норматив равным $\businesstripratesymbol~=~\businesstripratevalue$ получим величину расходов
\begin{equation}
	\businesstripchargessymbol = \frac{\basewagesymbol \cdot \businesstripratesymbol}{100\%} = \frac{\basewagevalue \cdot \businesstripratevalue}{100\%} = \businesstripchargesvalue~\byn
\end{equation}

Расходы по статье <<Прочие затраты>> включают затраты на приобретение и подготовку специальной научно-технической информации и специальной литературы. Определяются по нормативу в процентах к основной заработной плате. Принимая норматив равным $\otherchargesratesymbol~=~\otherchargesratevalue$ получим величину расходов
\begin{equation}
	\otherchargessymbol = \frac{\basewagesymbol \cdot \otherchargesratesymbol}{100\%} = \frac{\basewagevalue \cdot \otherchargesratevalue}{100\%} = \otherchargesvalue~\byn
\end{equation}

Затраты по статье <<Накладные расходы>>, связанные с необходимостью  содержания  аппарата  управления,  вспомогательных хозяйств и опытных (экспериментальных) производств, а также с расходами на общехозяйственные нужды, относятся к конкретному ПО по нормативу в процентном отношении к основной заработной плате
исполнителей. Принимая норматив равным $\overheadratesymbol~=~\overheadratevalue$ получим величину расходов
\begin{equation}
	\overheadchargessymbol = \frac{\basewagesymbol \cdot \overheadratesymbol}{100\%} = \frac{\basewagevalue \cdot \overheadratevalue}{100\%} = \overheadvalue~\byn
\end{equation}

Общая сумма расходов по смете определяется как сумма вышерассчитанных показателей

\begin{equation}
\begin{aligned}
	\totalchargessymbol = \basewagesymbol + \additionalwagesymbol + \ssfchargessymbol &+ \insurancechargessymbol + \consumableschargessymbol + \machinetimechargessymbol + \businesstripchargessymbol + \otherchargessymbol + \overheadchargessymbol =\\
	&= \totalchargesvalue~\byn
\end{aligned}
\end{equation}

Рентабельность определяется из результатов анализа рыночных условий и переговоров с потребителями ПО. Исходя из принятого уровня рентабельности $\profitabilityratesymbol~=~\profitabilityratevalue$, прибыль от реализации ПО составит
\begin{equation}
	\profitabilitysymbol = \frac{\totalchargessymbol \cdot \profitabilityratesymbol}{100\%} = \frac{\totalchargesvalue \cdot \profitabilityratevalue}{100\%} = \profitabilityvalue~\byn
\end{equation}

На основании расчета прибыли и уровня себестоимости рассчитаем прогнозируемую цену программного средства без учета налогов
\begin{equation}
	\netpricesymbol = \totalchargessymbol + \profitabilitysymbol = \totalchargesvalue + \profitabilityvalue = \netpricevalue~\byn
\end{equation}

Далее рассчитаем налог на добавленную стоимость
\begin{equation}
	\vatsymbol = \frac{\netpricesymbol \cdot \vatratesymbol}{100\%} = \frac{\netpricevalue \cdot \vatratevalue}{100\%} = \vatvalue~\byn
\end{equation}

НДС включается в прогнозируемую отпускную цену
\begin{equation}
	\sellingpricesymbol = \netpricesymbol + \vatsymbol = \netpricevalue + \vatvalue = \sellingpricevalue~\byn
\end{equation}

\subsection{Оценка экономического эффекта от продажи программного средства}
\label{sec:economics:effect}

Положительный экономический эффект для разработчика программного средства достигается за счет получения прибыли от реализация мобильного приложения большому количеству пользователей. Прибыль достигаетсяза счет реализации мобильного приложения на рынку мобильных приложений. Прибыль от реализации зависит от количества клиентов, ценны мобильного приложения и затрат на разработку.

Исходя из тенденций на рынке мобильных приложений и маркетингового исследования, приложение будет востребовано ны рынке в течение четырех лет: в 2019 году планируется реализовать \countfirstyearcopys~копий, в 2020 году --- \countsecondyearcopys~копий, 2021 год --- \countthirdyearcopys~копий и \counfoursyearcopys~в 2022 году.

На основании анализа цен на аналоничные приложения мы определим цену мобильного приложения: \oneCopyPrice~\byn

Прибыль от продажи одной копии рассчитывается по формуле:
\begin{equation}
	\label{eq:economics:estimate:onecopyprofit}
	\begin{aligned}
		\onecopyprofit &= \text{Ц} - \vatsymbola - \frac{\text{З}_\text{р}}{\countcopyssymbol},
	\end{aligned}
	\end{equation}
	\begin{explanation}
	где & $ \text{Ц} $ & цена реализации одной копии, \byn;\\
		& $ \vatsymbola $ & сумма налога на добавленную стоимость, \byn;\\
		& $ \text{З}_\text{р} $ & сумма расходов на разработку и реализацию, \byn;\\
		& $ \countcopyssymbol $ & количество реализованных копий.
	\end{explanation}

Сумма налога на добавленную стоимость
\begin{equation}
	\vatsymbola = \frac{\text{Ц} \cdot \%\vatsymbola}{100 + \%\vatsymbol} = \frac{\oneCopyPrice \cdot \vatrate}{100 + \vatrate} = \vatsymbolavalue~\byn
\end{equation}

 Подставим значения в формулу ~\ref{eq:economics:estimate:onecopyprofit}:
 \begin{equation}
	\onecopyprofit = \oneCopyPrice - \vatsymbolavalue - \frac{\sellingpricevalue}{\countcopys} = \onecopyprofitvalue~\byn
\end{equation}

Cуммарная прибыль за год рассчитывется как:
\begin{equation}
	\sumprofitperyearsymbol = \text{П}_\text{ед} \cdot \countcopyssymbol,
\end{equation}

Следовательно прибыль от программного продукта по годам составляет:
\begin{equation}
	\begin{aligned}
		\text{П}_\text{1} = \onecopyprofitvalue \cdot \countfirstyearcopys = \profitperyearvalueone~\byn,\\
		\text{П}_\text{2} = \onecopyprofitvalue \cdot \countsecondyearcopys = \profitperyearvaluetwo~\byn,\\
		\text{П}_\text{3} = \onecopyprofitvalue \cdot \countthirdyearcopys = \profitperyearvaluethree~\byn,\\
		\text{П}_\text{4} = \onecopyprofitvalue \cdot \counfoursyearcopys = \profitperyearvaluefour~\byn,\\
	\end{aligned}
\end{equation}

Чистая прибыль с учетом налога на прибыль будет рассчитываться по формуле:
\begin{equation}
	\begin{aligned}
		\clearProfitSymol &= \sumprofitperyearsymbol - \frac{\sumprofitperyearsymbol \cdot \incomTaxRateSymbol}{100},
	\end{aligned}
\end{equation}
\begin{explanation}
	где & $ \incomTaxRateSymbol $ & ставка налога на прибыль (в 2019 г. составляет \num{18}\%~\cite{incomeTaxRate}).
\end{explanation}
\vspace{-3mm}

Чистая прибыль по годам составит:
\begin{equation}
	\begin{aligned}
		\text{ЧП}_\text{1} &= \profitperyearvalueone - \profitperyearvalueone \cdot \incomTaxRate = \clearProfitValueOne~\byn,\\
		\text{ЧП}_\text{2} &= \profitperyearvaluetwo - \profitperyearvaluetwo \cdot \incomTaxRate = \clearProfitValueTwo~\byn,\\
		\text{ЧП}_\text{3} &= \profitperyearvaluethree - \profitperyearvaluethree \cdot \incomTaxRate = \clearProfitValueThree~\byn,\\
		\text{ЧП}_\text{4} &= \profitperyearvaluefour - \profitperyearvaluefour \cdot \incomTaxRate = \clearProfitValueFour~\byn
	\end{aligned}
\end{equation}

\subsection{Расчет показателей эффективности}
\label{sec:economics:effect}

В ходе реализации ПС чистая прибыль в конечном итоге возмещает расходы на разработку.
Однако полученные при этом суммы результатов (прибыли) и затрат по годам приводят к единому времени --- расчетному году (за расчетный год принят 2019 год) путем умножения результатов затрат за каждый год на коэффициент дисконтирования $\discontKSymbol$, который рассчитывается по формуле:
\begin{equation}
	\begin{aligned}
		\discontKSymbol &= \text{1 + E}^{\text{t}_{\text{p}} - t},
	\end{aligned}
\end{equation}
\begin{explanation}
	где & $ E $ & норматив приведения равновременных затрат и результатов;\\
		& $ \text{t}_{\text{p}} $ & номер года, результаты которого приводятся к расчетному;\\
		& $ t $ & номер расчетного года.
\end{explanation}

На 27.06.2018 ставка рефинансирования состовляет 10\%~\cite{refinancingTax}.
Итого коэффициент дисконтирования по годам (2019 - 2022):
\begin{equation}
	\begin{aligned}
		\text{\(\alpha\)}_\text{1} &= 1,\\
		\text{\(\alpha\)}_\text{2} &= \text{(1 + \refinancingTax)}^{\text{1 - 2}} = \discontKValueOne,\\
		\text{\(\alpha\)}_\text{3} &= \text{(1 + \refinancingTax)}^{\text{1 - 3}} = \discontKValueTwo,\\
		\text{\(\alpha\)}_\text{4} &= \text{(1 + \refinancingTax)}^{\text{1 - 4}} = \discontKValueThree.
	\end{aligned}
\end{equation}

Cведем данные расчета вэкономического эффекта в таблицу~\ref{table:economics:estimate:economEffect}.
\begin{table}[!ht]
  \caption{Расчет экономического эффекта}
  \label{table:economics:estimate:economEffect}
  \begin{tabular}{| >{\centering}m{0.3\textwidth} 
                  | >{\centering}m{0.107\textwidth} 
                  | >{\centering}m{0.107\textwidth} 
                  | >{\centering}m{0.107\textwidth} 
									| >{\centering}m{0.107\textwidth} 
                  | >{\centering\arraybackslash}m{0.1\textwidth}|}
\hline
		\multirow{2}{*}{Показатели} & \multirow{2}{*}{Ед. изм.} & \multicolumn{4}{c|}{Годы} \\ \cline{3-6}
		 & & 2019 & 2020 & 2021 & 2022 \\ 
\hline
\multicolumn{6}{|c|}{Результаты:} \\ \hline
\multicolumn{1}{|c|}{Экономический эффект} & руб. & \multicolumn{1}{c|}{\clearProfitValueOne} & \clearProfitValueTwo & \clearProfitValueThree & \profitperyearvaluefour \\ \hline
	\end{tabular}
\end{table}

Полученные результаты свидетельствуют об эффективности разработки проектируемого программного средства.

% Cведем данные расчета вэкономического эффекта в таблицу 7.2
% \begin{table}[!ht]
% 	\begin{tabularx}{\textwidth}{|c|c|c|c|c|c|}
% 	\hline
% 	\multirow{2}{*}{Показатели} & \multirow{2}{*}{Ед. изм.} & \multicolumn{4}{c|}{Годы} \\ \cline{3-6} 
% 	& & 2019 & \multicolumn{1}{c|}{2020} & \multicolumn{1}{c|}{2021} & \multicolumn{1}{c|}{2022} \\ \hline
% 	\multicolumn{6}{|c|}{Результаты:} \\ \hline
% 	\multicolumn{1}{|c|}{Экономический эффект} & руб. & \multicolumn{1}{c|}{} & & & \\ \hline
% 	\end{tabularx}
% 	\end{table}

% \begin{longtable}{{|>{\raggedright}m{0.32\textwidth} | 
% 					>{\centering}m{0.1375\textwidth} | 
% 					>{\centering}m{0.1375\textwidth} | 
% 					>{\centering}m{0.1375\textwidth} | 
% 					>{\centering\arraybackslash}m{0.1375\textwidth}|}}
% \caption{Расчет экономического эффекта от использования нового ПО}
% \label{table:economics:effect:final_data}
% \centering

%   	\hline
% 	\begin{minipage}{1\linewidth}
% 		\centering Название показателя
% 	\end{minipage} & 
% 	\begin{minipage}{1\linewidth}
% 		\centering 2017 г.
% 	\end{minipage} & 
% 	\begin{minipage}{1\linewidth}
% 		\centering 2018 г.
% 	\end{minipage} & 
% 	\begin{minipage}{1\linewidth}
% 		\centering 2019 г.
% 	\end{minipage} & 
% 	\begin{minipage}{1\linewidth}
% 		\centering\arraybackslash 2020 г.
% 	\end{minipage} \endfirsthead 
% 	\caption*{Продолжение таблицы \ref{table:economics:effect:final_data}}\\\hline
% 	\centering 1 & \centering 2 & \centering 3 & \centering 4 & \centering\arraybackslash 5 \\\hline \endhead

% 	\hline
% 	\centering 1 & \centering 2 & \centering 3 & \centering 4 & \centering\arraybackslash 5 \\

% 	\hline
% 	\emph{Результаты} & & & & \\

% 	\hline
% 	Коэффициент приведения & \num{1} & \num{0.8696} & \num{0.7561} & \num{0.6575} \\

% 	\hline
% 	Прирост прибыли за счет экономии затрат ($\usernetprofitsymbol$),~\byn & --- & \usernetprofitvalue & \usernetprofitvalue & \usernetprofitvalue \\

% 	\hline
% 	То же с учетом фактора времени,~\byn  & --- & \usernetprofityearonevalue & \usernetprofityeartwovalue & \usernetprofityearthreevalue \\

% 	\hline
% 	\emph{Затраты} & & & & \\

% 	\hline
% 	Приобретение ПО ($\purchasecostsymbol$),~\byn & \sellingpricevalue & --- & --- & --- \\

% 	\hline
% 	Освоение ПО ($\deploymentcostsymbol$),~\byn & \deploymentchargesvalue & --- & --- & --- \\

% 	\hline
% 	Сопровождение ПО ($\maintenancecostsymbol$),~\byn & \maintenancechargesvalue & --- & --- & --- \\

% 	Всего затрат ($\capitalinvestmentsymbol$),~\byn & \capitalinvestmentvalue & --- & --- & --- \\

% 	\hline
% 	То же с учетом фактора времени,~\byn & \capitalinvestmentvalue & --- & --- & --- \\

% 	\hline
% 	\emph{Экономический эффект} & & & & \\

% 	\hline
% 	Превышение результата над затратами,~\byn & \excessovercostsyearzerovalue & \excessovercostsyearonevalue & \excessovercostsyeartwovalue & \excessovercostsyearthreevalue \\

% 	\hline
% 	То же с нарастающим итогом,~\byn & \excessovercostsyearzerovalue & \excessovercostswithtimingyearonevalue & \excessovercostswithtimingyeartwovalue & \excessovercostswithtimingyearthreevalue \\

% 	\hline
% \end{longtable}