\subsection{Требования к проектируемому программному средству}
\label{sec:analysis:specification}

По результатам изучения предметной области, анализа литературных источников и обзора существующих систем-аналогов сформулируем требования к проектируемому программному средству.

\subsubsection{} Назначение проекта
\label{sec:analysis:specification:purpose}

Назначением проекта является разработка программного средства, представляющего специализированную (тематическую) социальную сеть для мотоциклистов.

\subsubsection{} Основные функции
\label{sec:analysis:specification:functions}

Программное средство должно поддерживать следующие основные фун\-к\-ции:

\begin{itemize}
	\item регистрация и аутентификация;
	\item управление друзьями и подписчиками;
	\item отправка сообщений;
	\item просмотр профиля пользователя;
	\item возможность создания и поиска публикаций, содержащих текст, фото и видеоматериалы;
	\item возможность добавления хештегов к описаниям публикаций;
	\item ведение личного блога;
	\item прикрепления информации о мотопарке пользователя к его профилю;
	\item установка статуса, уведомляющего о передвижение на мотоцикле в данными момент времени;
	\item просмотр карты с геолокацией,
	\item установка на карте меток предупрежения;
	\item подача сигнала <<SOS>>;
	\item отправка сигнала, представляющего собой запрос на совместную поездку (маячок); 
	\item отправка push-уведомлений.
\end{itemize}

\subsubsection{} Выбор технологий программирования
\label{sec:analysis:specification:language}

На данный момент на рынке мобильных устройств выделяются 2 основные платформы (операционные системы): Android и IOS. Разрабатываемое мобильное приложение должно работать на обеих платформах, при этом стоит цель минимизировать время разработки продукта без потерь его качества. До недавнего времени приходилось разрабатывать 2 отдельных приложения, копируя до 90 \% общего кода. Однако всё больше внимания привлекают кроссплатформенные решения, которые базируются на применении кроссплатформенных сред разработки. В связи с этим а основу клиентскойй части приложения был выбран React Native.

Для вхождения в разработку с помощью React-Native, нет необходимости осваивать новые языки, если JavaScript, HTML, CSS уже изучены, а программирование напоминает
создание веб интерфейса для веб версий. Инструменты и архитектура приложения такие же, как и при в веб-разработке. 

После анализа возможных архитекрутрных решение (\ref{sec:analysis:literature:architecture}) было приянято решение использовать шаблон проектирования Viper.

Viper превосходит все представленные архитектурные паттерны в
распределённой, тестируемости, и простоте освоения новых модулей.
Основными особенностями Viper является распространение – VIPER
отличается отличной способностью к распределению обязанностей,
тестируемость – при лучшем распределении, получается лучшая тестируемость.
Для наилучшего распределения обязанностей,
и покрытия приложения тестами приложения, больше всего подходит
архитектурный шаблон -- Viper. 

Сформулированные требования позволят осуществить успешное проек- тирование и разработку программного средства.

