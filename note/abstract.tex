\sectioncentered*{Реферат}
\thispagestyle{empty}

% Зачем: чтобы можно было вывести общее число страниц.
% Добавляется единица, поскольку последняя страница -- ведомость.
\FPeval{\totalpages}{round(\getpagerefnumber{LastPage} + 1, 0)}

\noindent
{БГУИР ДП 1-40 05 01-02 051 ПЗ}\\

\textbf{Покосенко В.Ю.} Тематические социальные сети и кроссплатформенное мобильное приложения для мотоциклистов : пояснительная записка к дипломному проекту / В.Ю. Покосенко. – Минск : БГУИР, 2019. – \num{\totalpages}~с. \\

Пояснительная записка \num{\totalpages}~с., \num{\totfig{}}~рис., \num{\tottab{}}~табл., \num{\toteq{}}~формул, \num{\totref{}}~источника.

\emph{Ключевые слова}: тематические социальные сети, кроссплатформенность, мобильное приложение. \\

Целью данного дипломного проекта является создание комфортных условий для построения взаимоотношений пользователей и их коммунникации в социальной сети (на примере сообщества мотоциклистов) путем создания мобильного приложения. 
Оно позволит людям общаться между собой, находить новые
знакомства, совершать совместные поездки, искать интересующую информацию, вести свой личный блог.

В процессе анализа предметной области были выделены основные аспекты функционирования социальной сети. Выработаны функциональные и нефункциональные требования.

Была разработана архитектура программной системы, для каждой ее составной части было проведено разграничение реализуемых задач проектирование, уточнение используемых технологий и собственно разработка. Были выбраны наиболее современные средства разработки, широко применяемые в индустрии. 

Полученные в ходе технико-экономического обоснования результаты о прибыли для разработчика, пользователя, уровень рентабельности, а также экономический эффект доказывают целесообразность разработки про\-екта.

