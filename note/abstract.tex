\sectioncentered*{Реферат}
\thispagestyle{empty}

% Зачем: чтобы можно было вывести общее число страниц.
% Добавляется единица, поскольку последняя страница -- ведомость.
\FPeval{\totalpages}{round(\getpagerefnumber{LastPage} + 1, 0)}

\begin{center}
	Пояснительная записка \num{\totalpages}~с., \num{\totfig{}}~рис., \num{\tottab{}}~табл., \num{\toteq{}}~формул, \num{\totref{}}~источника.
	\MakeUppercase{Программное средство, мобильное приложение, социальная сеть, кроссплатформенность, мотоциклы, публикация}
\end{center}

Целью данного дипломного проекта является разработка программного средства 
для упрощения коммуникации больших групп пользователей,
интересующихся определённой предметной областью, в данном случае --
мотоциклами. Оно позволит людям общаться между собой, находить новые
знакомства, искать интересующую информацию, вести свой личный блог.

В процессе анализа предметной области были выделены основные аспекты функционирования социальной сети. Выработаны функциональные и нефункциональные требования.

Была разработана архитектура программной системы, для каждой ее составной части было проведено разграничение реализуемых задач проектирование, уточнение используемых технологий и собственно разработка. Были выбраны наиболее современные средства разработки, широко применяемые в индустрии. 

Полученные в ходе технико-экономического обоснования результаты о прибыли для разработчика, пользователя, уровень рентабельности, а также экономический эффект доказывают целесообразность разработки про\-екта.

